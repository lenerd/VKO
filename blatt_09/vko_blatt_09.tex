\documentclass[a4paper]{scrartcl}

% font/encoding packages
\usepackage[utf8]{inputenc}
\usepackage[T1]{fontenc}
\usepackage{lmodern}
\usepackage[ngerman]{babel}
\usepackage[ngerman=ngerman-x-latest]{hyphsubst}

\usepackage{amsmath, amssymb, amsfonts, amsthm}
\usepackage{mathtools}
\usepackage{array}
\usepackage{stmaryrd}
\usepackage{marvosym}
\usepackage{subcaption}
\allowdisplaybreaks{}
\usepackage[output-decimal-marker={,}]{siunitx}
\usepackage[shortlabels]{enumitem}
\usepackage[section]{placeins}
\usepackage{float}
\usepackage{units}
\usepackage{listings}
\usepackage{pgfplots}
\pgfplotsset{compat=1.12}
\usepackage{tikz}
\usepackage{tikz-qtree}
\usetikzlibrary{arrows,automata}
\usepackage{pdflscape}
\usepackage{xcolor}
\definecolor{light-gray}{HTML}{cccccc}
\usepackage{algorithm}
\usepackage{algpseudocode}
\usepackage{hhline}

\newcommand{\approxalg}[1]{$#1$-Ap\-pro\-xi\-ma\-ti\-ons\-al\-go\-rith\-mus}

\newtheorem*{proposition}{Behauptung}
\newtheorem*{definition}{Definition}
\newcommand{\gdw}{\ \Leftrightarrow\ }
\newcommand{\N}{\mathbb{N}}
\newcommand{\Oh}{\mathcal{O}}
\DeclareMathOperator{\im}{im}

\usepackage{fancyhdr}
\pagestyle{fancy}

\lstset{%
    frame=single,
    numbers=left,
    keepspaces,
    language=R,
    title=Listing: \lstname,
}

\def \blattnr {9}

\lhead{VKO -- Blatt {\blattnr}}
\rhead{Alina Bombeck, Lennart Braun, Carolin Konietzny, Tronje Krabbe}
\cfoot{\thepage}


\title{Vertiefung Kombinatorische Optimierung}
\subtitle{Blatt {\blattnr} Hausaufgaben}
\author{%
    Alina Bombeck, 6535392 (Gruppe 1) \and
    Lennart Braun, 6523742 (Gruppe 1) \and
    Carolin Konietzny, 6523939 (Gruppe 1) \and
    Tronje Krabbe, 6435002 (Gruppe 3)
}
\date{zum 27. Juni 2016}
\usepackage{pdfpages}

\begin{document}
\maketitle


\begin{enumerate}[label=\bfseries \arabic*.]
\item % 1.
\begin{enumerate}
    \item % a)
        TODO

    \item % b)
        TODO

    \item % c)
        Wir betrachten folgende Konfiguration:
        \begin{align*}
            t_1 = t_2 = 5\\
            t_3 = t_4 = 3
        \end{align*}
        Am Anfang werden die Jobs `zufällig' verteilt:
        \begin{align*}
            A& = \{1,2\} \Rightarrow T_1 = 10\\
            B& = \{3,4\} \Rightarrow T_2 =  6\\
            \Rightarrow& |T_1 - T_2| = 4
        \end{align*}
        Diese Aufteilung ist durch Verschieben eines Jobs nicht verbesserbar;
        die optimale Aufteilung wäre allerdings:
        \begin{align*}
            A& = \{1,3\} \Rightarrow T_1 = 8\\
            B& = \{2,4\} \Rightarrow T_2 = 8\\
            \Rightarrow& |T_1 - T_2| = 0
        \end{align*}
        \qed
\end{enumerate}

\item % 2.
    \underline{1. Iteration:}
    \\
    Profit-Tabelle:
    \\
    \begin{tabular}{c|c c c c}
          & e & f & g & h \\
        \hline
        a & 2 & 3 & 3 & 1 \\
        b & 4 & 3 & 1 & 1 \\
        c & 2 & 3 & 3 & 1 \\
        d & 3 & 4 & 0 & 0 \\
    \end{tabular}
    \\
    Hieraus entnehmen wir, dass entweder $b$ und $e$, oder $d$ und $f$ getauscht werden sollten. Wir entscheiden uns für
    $b$ und $e$ und erhalten den neuen Wert $w(A_1, B_1) = 7 - 4 = 3$, und die neue Aufteilung:
    \\
    \begin{tikzpicture}
        \node[label=180:a,shape=circle,fill=black,minimum size=0.1cm] (a) at (0, 3) {};
        \node[label=0:b,shape=circle,fill=black,minimum size=0.1cm] (b) at (3, 3) {};
        \node[label=180:c,shape=circle,fill=black,minimum size=0.1cm] (c) at (0, 1) {};
        \node[label=180:d,shape=circle,fill=black,minimum size=0.1cm] (d) at (0, 0) {};
        \node[label=180:e,shape=circle,fill=black,minimum size=0.1cm] (e) at (0, 2) {};
        \node[label=0:f,shape=circle,fill=black,minimum size=0.1cm] (f) at (3, 2) {};
        \node[label=0:g,shape=circle,fill=black,minimum size=0.1cm] (g) at (3, 1) {};
        \node[label=0:h,shape=circle,fill=black,minimum size=0.1cm] (h) at (3, 0) {};

        \draw (a) -- (e);
        \draw (a) -- (f);
        \draw (b) -- (f);
        \draw (b) .. controls(3.3,2) .. (g);
        \draw (c) -- (e);
        \draw (c) -- (f);
        \draw (d) -- (g);
        \draw (g) -- (h);
    \end{tikzpicture}

    \underline{2. Iteration}
    \\
    Profit-Tabelle:
    \\
    \begin{tabular}{c|c c c}
          & f & g & h \\
        \hline
        a & -1 & -1 & -1 \\
        c & -1 & -1 & -1 \\
        d & 2 & -2 & 0 \\
    \end{tabular}
    \\
    Hier ist leicht zu erkennen, dass wir $d$ und $f$ tauschen sollten, für einen
    verbesserten Wert von $w(A_2, B_2) = 3 - 2 = 1$, und eine neue Aufteilung:
    \\
    \begin{tikzpicture}
        \node[label=180:a,shape=circle,fill=black,minimum size=0.1cm] (a) at (0, 3) {};
        \node[label=0:b,shape=circle,fill=black,minimum size=0.1cm] (b) at (3, 3) {};
        \node[label=180:c,shape=circle,fill=black,minimum size=0.1cm] (c) at (0, 1) {};
        \node[label=0:d,shape=circle,fill=black,minimum size=0.1cm] (d) at (3, 2) {};
        \node[label=180:e,shape=circle,fill=black,minimum size=0.1cm] (e) at (0, 2) {};
        \node[label=180:f,shape=circle,fill=black,minimum size=0.1cm] (f) at (0, 0) {};
        \node[label=0:g,shape=circle,fill=black,minimum size=0.1cm] (g) at (3, 1) {};
        \node[label=0:h,shape=circle,fill=black,minimum size=0.1cm] (h) at (3, 0) {};

        \draw (a) -- (e);
        \draw (a) .. controls(0.4,1.5) .. (f);
        \draw (b) -- (f);
        \draw (b) .. controls(3.3,2) .. (g);
        \draw (c) -- (e);
        \draw (c) -- (f);
        \draw (d) -- (g);
        \draw (g) -- (h);
    \end{tikzpicture}
    
    \underline{3. Iteration}
    \\
    Profit-Tabelle:
    \\
    \begin{tabular}{c|c c}
          & g & h \\
        \hline
        a & -5 & -3 \\
        c & -5 & -3 \\
    \end{tabular}
    \\
    Da hier nur negative Werte vorhanden sind, ist keine weitere Verbesserung
    möglich.
\end{enumerate}
\end{document}
