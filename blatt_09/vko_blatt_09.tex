\documentclass[a4paper]{scrartcl}

% font/encoding packages
\usepackage[utf8]{inputenc}
\usepackage[T1]{fontenc}
\usepackage{lmodern}
\usepackage[ngerman]{babel}
\usepackage[ngerman=ngerman-x-latest]{hyphsubst}

\usepackage{amsmath, amssymb, amsfonts, amsthm}
\usepackage{mathtools}
\usepackage{array}
\usepackage{stmaryrd}
\usepackage{marvosym}
\usepackage{subcaption}
\allowdisplaybreaks{}
\usepackage[output-decimal-marker={,}]{siunitx}
\usepackage[shortlabels]{enumitem}
\usepackage[section]{placeins}
\usepackage{float}
\usepackage{units}
\usepackage{listings}
\usepackage{pgfplots}
\pgfplotsset{compat=1.12}
\usepackage{tikz}
\usepackage{tikz-qtree}
\usetikzlibrary{arrows,automata}
\usepackage{pdflscape}
\usepackage{xcolor}
\definecolor{light-gray}{HTML}{cccccc}
\usepackage{algorithm}
\usepackage{algpseudocode}
\usepackage{hhline}

\newcommand{\approxalg}[1]{$#1$-Ap\-pro\-xi\-ma\-ti\-ons\-al\-go\-rith\-mus}

\newtheorem*{proposition}{Behauptung}
\newtheorem*{definition}{Definition}
\newcommand{\gdw}{\ \Leftrightarrow\ }
\newcommand{\N}{\mathbb{N}}
\newcommand{\Oh}{\mathcal{O}}
\DeclareMathOperator{\im}{im}

\usepackage{fancyhdr}
\pagestyle{fancy}

\lstset{%
    frame=single,
    numbers=left,
    keepspaces,
    language=R,
    title=Listing: \lstname,
}

\def \blattnr {9}

\lhead{VKO -- Blatt {\blattnr}}
\rhead{Alina Bombeck, Lennart Braun, Carolin Konietzny, Tronje Krabbe}
\cfoot{\thepage}


\title{Vertiefung Kombinatorische Optimierung}
\subtitle{Blatt {\blattnr} Hausaufgaben}
\author{%
    Alina Bombeck, 6535392 (Gruppe 1) \and
    Lennart Braun, 6523742 (Gruppe 1) \and
    Carolin Konietzny, 6523939 (Gruppe 1) \and
    Tronje Krabbe, 6435002 (Gruppe 3)
}
\date{zum 27. Juni 2016}
\usepackage{pdfpages}

\begin{document}
\maketitle


\begin{enumerate}[label=\bfseries \arabic*.]
\item % 1.
\begin{enumerate}
    \item % a)
        Es gelte $t_j \leq \frac{1}{2} \sum_{i=1}^n t_i$ für alle Jobs $j$ $(\star)$.
        \begin{proposition}
            Für jede lokal optimale Lösung gilt
            \begin{equation*}
                \frac{1}{2}T_1 \leq T_2 \leq 2T_1.
            \end{equation*}
        \end{proposition}
        \begin{proof}
            Bei $T_1 = T_2$ ist der Fall klar.
            Gelte nun $T_1 < T_2$ o.\,B.\,d.\,A.
            Dann bleibt zu zeigen, dass $T_2 \leq 2T_1$ gilt.
            Nach $(\star)$ besteht $T_2$ aus mindestens zwei Jobs.
            Sei $t_j$ minimal in $T_2$. Dann gilt $t_j \leq \frac{1}{2}T_2$.

            Da es sich um ein lokales Optimum handelt, muss weiterhin
            $t_j \geq |T_1 - T_2| = T_2 - T_1$ gelten, denn sonst könnte durch
            Verschieben von $t_j$ eine bessere Lösung erziehlt werden.
            Es folgt
            \begin{equation*}
                t_j \geq T_2 - T_1
                \gdw T_2 \leq T_1 + t_j \leq T_1 + \frac{1}{2}T_2
                \ \Rightarrow\ \frac{1}{2}T_2 \leq T_1
                \gdw T_2 \leq 2T_1.
                \qedhere
            \end{equation*}
        \end{proof}

    \item % b)
        Ein Job $t_j$ kann nur verschoben werden, wenn $t_j < |T_1 - T_2|$ gilt.
        Gelte $T_1 > T_2$.
        Dann wird der größte Job $t_j$ aus $T_1$, der $|T_1 - T_2|$ verbessert,
        nach $T_2$ verschoben.
        Sei dies das erste Mal, dass $t_j$ verschoben wird.

        Ist ein lokales Optimum erreicht, terminiert der Algorithmus und $t_j$
        wird nicht mehrmals verschoben.
        Andernfalls
        \begin{itemize}
            \item $T_1 < T_2$: \\
                Es muss nun $|T_1-T_2| = T_2 - T_1 < t_j$ sein, da die
                Differenz verkleinert wurde.
                $t_j$ kann also nicht wieder verschoben werden.
                Da sich während des Algorithmus die Differenz stehts
                verkleinert, wird $t_j$ auch in Zukunft nicht noch ein weiteres
                Mal verschoben.

            \item $T_1 > T_2$: \\
                Es werden weitere Jobs von $T_1$ nach $T_2$ geschoben.
                Sei $t_k$ der letzte Job, der verschoben wird, solange
                $T_1 > T_2$.
                Nach der Auswahlvorschrift gilt $t_k \leq t_j$.
                Nach dem Verschieben von $t_k$ gilt wie im ersten Fall
                $T_1 < T_2$, dass $t_k$ nicht wieder verschoben werden kann.
                Da aber $t_j \geq t_k$ gilt dies ebenfalls für $t_j$.
        \end{itemize}
        Ergo werden einmal verschobene Jobs nicht wieder zurückgeschoben.

        $\Rightarrow$ „Bei $n$ Jobs sind höchstens $n$ Iterationen möglich.“

    \item % c)
        Wir betrachten folgende Konfiguration:
        \begin{equation*}
            t_1 = t_2 = 5,
            \qquad\qquad
            t_3 = t_4 = 3
        \end{equation*}
        Am Anfang werden die Jobs „zufällig“ verteilt:
        \begin{equation*}
            \begin{aligned}
                A = \{1,2\} &\Rightarrow T_1 = 10 \\
                B = \{3,4\} &\Rightarrow T_2 =  6
            \end{aligned}
            \quad \text{mit }
            |T_1 - T_2| = 4
        \end{equation*}
        Diese Aufteilung ist durch Verschieben eines Jobs nicht verbesserbar;
        die optimale Aufteilung wäre allerdings:
        \begin{equation*}
            \begin{aligned}
                A = \{1,3\} &\Rightarrow T_1 = 8 \\
                B = \{2,4\} &\Rightarrow T_2 = 8
            \end{aligned}
            \quad \text{mit }
            |T_1 - T_2| = 0
        \end{equation*}
\end{enumerate}

\item % 2.
    Gegeben ist der folgende Graph.
    \begin{figure}[H]
        \centering
        \begin{tikzpicture}
            \node[label=180:a,shape=circle,fill=black,minimum size=0.1cm] (a) at (0, 3) {};
            \node[label=180:b,shape=circle,fill=black,minimum size=0.1cm] (b) at (0, 2) {};
            \node[label=180:c,shape=circle,fill=black,minimum size=0.1cm] (c) at (0, 1) {};
            \node[label=180:d,shape=circle,fill=black,minimum size=0.1cm] (d) at (0, 0) {};
            \node[label=0:e,shape=circle,fill=black,minimum size=0.1cm] (e) at (3, 3) {};
            \node[label=0:f,shape=circle,fill=black,minimum size=0.1cm] (f) at (3, 2) {};
            \node[label=0:g,shape=circle,fill=black,minimum size=0.1cm] (g) at (3, 1) {};
            \node[label=0:h,shape=circle,fill=black,minimum size=0.1cm] (h) at (3, 0) {};

            \node (A) at (0, -1) {$A$};
            \node (B) at (3, -1) {$B$};

            \draw (a) -- (e);
            \draw (a) -- (f);
            \draw (b) -- (f);
            \draw (b) -- (g);
            \draw (c) -- (e);
            \draw (c) -- (f);
            \draw (d) -- (g);
            \draw (g) -- (h);
        \end{tikzpicture}
    \end{figure}

    \underline{Iteration 1}

    Profit-Tabelle:
    \begin{table}[H]
        \centering
        \begin{tabular}{c|c c c c}
              & e & f & g & h \\
            \hline
            a & 2 & 3 & 3 & 1 \\
            b & 4 & 3 & 1 & 1 \\
            c & 2 & 3 & 3 & 1 \\
            d & 3 & 4 & 0 & 0 \\
        \end{tabular}
    \end{table}
    Hieraus entnehmen wir, dass entweder $b$ und $e$, oder $d$ und $f$
    getauscht werden sollten. Wir entscheiden uns für $b$ und $e$ und erhalten
    den Schnitt $(A_1, B_1)$ mit Gewicht $w(A_1, B_1) = 7 - 4 = 3$.
    \begin{figure}[H]
        \centering
        \begin{tikzpicture}
            \node[label=180:a,shape=circle,fill=black,minimum size=0.1cm] (a) at (0, 3) {};
            \node[label=0:b,shape=circle,fill=black,minimum size=0.1cm] (b) at (3, 3) {};
            \node[label=180:c,shape=circle,fill=black,minimum size=0.1cm] (c) at (0, 1) {};
            \node[label=180:d,shape=circle,fill=black,minimum size=0.1cm] (d) at (0, 0) {};
            \node[label=180:e,shape=circle,fill=black,minimum size=0.1cm] (e) at (0, 2) {};
            \node[label=0:f,shape=circle,fill=black,minimum size=0.1cm] (f) at (3, 2) {};
            \node[label=0:g,shape=circle,fill=black,minimum size=0.1cm] (g) at (3, 1) {};
            \node[label=0:h,shape=circle,fill=black,minimum size=0.1cm] (h) at (3, 0) {};

            \node[circle, draw, minimum size=0.5cm] at (b) {};
            \node[circle, draw, minimum size=0.5cm] at (e) {};

            \node (A) at (0, -1) {$A_1$};
            \node (B) at (3, -1) {$B_1$};

            \draw (a) -- (e);
            \draw (a) -- (f);
            \draw (b) -- (f);
            \draw [bend angle=22, bend left] (b) to (g);
            \draw (c) -- (e);
            \draw (c) -- (f);
            \draw (d) -- (g);
            \draw (g) -- (h);
        \end{tikzpicture}
    \end{figure}

    \underline{Iteration 2}

    Profit-Tabelle:
    \begin{table}[H]
        \centering
        \begin{tabular}{c|c c c}
              & f & g & h \\
            \hline
            a & -1 & -1 & -1 \\
            c & -1 & -1 & -1 \\
            d & 2 & -2 & 0 \\
        \end{tabular}
    \end{table}
    Hier ist leicht zu erkennen, dass wir $d$ und $f$ tauschen sollten, für einen
    Schnitt $(A_2, B_2)$ mit Gewicht von $w(A_2, B_2) = 3 - 2 = 1$.
    \begin{figure}[H]
        \centering
        \begin{tikzpicture}
            \node[label=180:a,shape=circle,fill=black,minimum size=0.1cm] (a) at (0, 3) {};
            \node[label=0:b,shape=circle,fill=black,minimum size=0.1cm] (b) at (3, 3) {};
            \node[label=180:c,shape=circle,fill=black,minimum size=0.1cm] (c) at (0, 1) {};
            \node[label=0:d,shape=circle,fill=black,minimum size=0.1cm] (d) at (3, 2) {};
            \node[label=180:e,shape=circle,fill=black,minimum size=0.1cm] (e) at (0, 2) {};
            \node[label=180:f,shape=circle,fill=black,minimum size=0.1cm] (f) at (0, 0) {};
            \node[label=0:g,shape=circle,fill=black,minimum size=0.1cm] (g) at (3, 1) {};
            \node[label=0:h,shape=circle,fill=black,minimum size=0.1cm] (h) at (3, 0) {};

            \node[circle, draw, minimum size=0.5cm] at (b) {};
            \node[circle, draw, minimum size=0.5cm] at (e) {};
            \node[circle, draw, minimum size=0.5cm] at (d) {};
            \node[circle, draw, minimum size=0.5cm] at (f) {};

            \node (A) at (0, -1) {$A_2$};
            \node (B) at (3, -1) {$B_2$};

            \draw (a) -- (e);
            \draw [bend left] (a) to (f);
            \draw (b) -- (f);
            \draw [bend right] (b) to (g);
            \draw (c) -- (e);
            \draw (c) -- (f);
            \draw (d) -- (g);
            \draw (g) -- (h);
        \end{tikzpicture}
    \end{figure}

    \underline{3. Iteration}

    Profit-Tabelle:
    \begin{table}[H]
        \centering
        \begin{tabular}{c|c c}
              & g & h \\
            \hline
            a & -5 & -3 \\
            c & -5 & -3 \\
        \end{tabular}
    \end{table}
    Die Paare $a, h$ und $c, h$ können getauscht werden, da die
    Verschlechterung am geringsten ist. Wir entscheiden uns für ersteres und
    erhalten einen Schnitt $(A_3, B_3)$ mit Gewicht von
    $w(A_3, B_3) = 1 - (-3) = 4$.
    \begin{figure}[H]
        \centering
        \begin{tikzpicture}
            \node[label=0:a,shape=circle,fill=black,minimum size=0.1cm] (a) at (3, 0) {};
            \node[label=0:b,shape=circle,fill=black,minimum size=0.1cm] (b) at (3, 3) {};
            \node[label=180:c,shape=circle,fill=black,minimum size=0.1cm] (c) at (0, 1) {};
            \node[label=0:d,shape=circle,fill=black,minimum size=0.1cm] (d) at (3, 2) {};
            \node[label=180:e,shape=circle,fill=black,minimum size=0.1cm] (e) at (0, 2) {};
            \node[label=180:f,shape=circle,fill=black,minimum size=0.1cm] (f) at (0, 0) {};
            \node[label=0:g,shape=circle,fill=black,minimum size=0.1cm] (g) at (3, 1) {};
            \node[label=180:h,shape=circle,fill=black,minimum size=0.1cm] (h) at (0, 3) {};

            \node[circle, draw, minimum size=0.5cm] at (b) {};
            \node[circle, draw, minimum size=0.5cm] at (e) {};
            \node[circle, draw, minimum size=0.5cm] at (d) {};
            \node[circle, draw, minimum size=0.5cm] at (f) {};
            \node[circle, draw, minimum size=0.5cm] at (a) {};
            \node[circle, draw, minimum size=0.5cm] at (h) {};

            \node (A) at (0, -1) {$A_3$};
            \node (B) at (3, -1) {$B_3$};

            \draw (a) -- (e);
            \draw (a) to (f);
            \draw (b) -- (f);
            \draw [bend right] (b) to (g);
            \draw (c) -- (e);
            \draw (c) -- (f);
            \draw (d) -- (g);
            \draw (g) -- (h);
        \end{tikzpicture}
    \end{figure}

    Die KL-Nachbarschaft von $(A,B)$ besteht aus den Schnitten $(A_i, B_i)$,
    $i=1,2,3$.
    Wir würden nun mit $(A_2, B_2)$ fortfahren, jedoch ist zu sehen, dass es
    keinen besseren Schnitt gibt: Der Graph ist zusammenhängend und der
    gewählte Schnitt besteht aus einer einzelnen Kante.


\end{enumerate}
\end{document}
