\documentclass[a4paper]{scrartcl}

\usepackage{scrlayer-scrpage}

% font/encoding packages
\usepackage[utf8]{inputenc}
\usepackage[T1]{fontenc}
\usepackage{lmodern}
\usepackage[ngerman]{babel}
\usepackage[ngerman=ngerman-x-latest]{hyphsubst}

\usepackage{amsmath, amssymb, amsfonts, amsthm}
\usepackage{mathtools}
\usepackage{array}
\usepackage{stmaryrd}
\usepackage{marvosym}
\usepackage{subcaption}
\allowdisplaybreaks{}
\usepackage[output-decimal-marker={,}]{siunitx}
\usepackage[shortlabels]{enumitem}
\usepackage[section]{placeins}
\usepackage{float}
\usepackage{units}
\usepackage{listings}
\usepackage{pgfplots}
\pgfplotsset{compat=1.12}
\usepackage{tikz}
\usepackage{tikz-qtree}
\usetikzlibrary{arrows,automata}
\usepackage{pdflscape}
\usepackage{xcolor}
\definecolor{light-gray}{HTML}{cccccc}
\usepackage{algorithm}
\usepackage{algpseudocode}
\usepackage{hhline}

\newcommand{\approxalg}[1]{$#1$-Ap\-pro\-xi\-ma\-ti\-ons\-al\-go\-rith\-mus}

\newtheorem*{proposition}{Behauptung}
\newtheorem*{definition}{Definition}
\newcommand{\gdw}{\ \Leftrightarrow\ }
\newcommand{\N}{\mathbb{N}}
\newcommand{\Oh}{\mathcal{O}}
\DeclareMathOperator{\im}{im}
\DeclareMathOperator{\Pr}{Pr}
\DeclareMathOperator{\E}{E}

\lstset{%
    frame=single,
    numbers=left,
    keepspaces,
    language=R,
    title=Listing: \lstname,
}

\def \blattnr {10}

\lohead{VKO -- Blatt {\blattnr}}
\rohead{Alina Bombeck, Lennart Braun, Carolin Konietzny, Tronje Krabbe}
\cofoot{\pagemark}
\setkomafont{pageheadfoot}{\textrm}
\KOMAoption{headsepline}{0.5pt:headsepline}
\pagestyle{scrheadings}


\title{Vertiefung Kombinatorische Optimierung}
\subtitle{Blatt {\blattnr} Hausaufgaben}
\author{%
    Alina Bombeck, 6535392 (Gruppe 1) \and
    Lennart Braun, 6523742 (Gruppe 1) \and
    Carolin Konietzny, 6523939 (Gruppe 1) \and
    Tronje Krabbe, 6435002 (Gruppe 3)
}
\date{zum 4. Juli 2016}
\usepackage{pdfpages}

\begin{document}
\maketitle


\begin{enumerate}[label=\bfseries \arabic*.]
\item % 1.
    Seien $d = \num{80000}$, $r = \num{20000}$.
    $X$ gebe die Anzahl der Stimmen für $D$ an.
    $X_i$ sei $1$, wenn Wähler $i$ für $D$ stimmt, sonst $0$.
    Die ersten $d$ Wähler sind Anhänger von $D$, die folgenden $r$ von $R$.
    Dann ist
    \begin{equation*}
        X = \sum_{i=1}^{d+r} X_i.
    \end{equation*}
    Nach Aufgabenstellung gilt
    \begin{equation*}
        \Pr[X_i = 1] =
        \begin{cases}
            \frac{99}{100}, 1 \leq i \leq d \\
            \frac{1}{100}, d+1 \leq i \leq d+r. \\
        \end{cases}
    \end{equation*}
    Mit der Linearität des Erwartungswertes folgt
    \begin{equation*}
        \begin{gathered}
            \E[X]
            = \E\left[ \sum_{i=1}^{d+r} X_i \right]
            = \sum_{i=1}^{d+r} \E[X_i]
            = \sum_{i=1}^{d+r} \Pr[X_i = 1] \\
            = \sum_{i=1}^{d} \frac{99}{100} + \sum_{i=d+1}^{d+r} \frac{1}{100}
            = d \cdot \frac{99}{100} + r \cdot \frac{1}{100}
            = \num{79400}.
        \end{gathered}
    \end{equation*}


\item % 2.
    \begin{enumerate}
        \item
        \item
    \end{enumerate}


\end{enumerate}
\end{document}
