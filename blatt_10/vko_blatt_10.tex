\documentclass[a4paper]{scrartcl}

\usepackage{scrlayer-scrpage}

% font/encoding packages
\usepackage[utf8]{inputenc}
\usepackage[T1]{fontenc}
\usepackage{lmodern}
\usepackage[ngerman]{babel}
\usepackage[ngerman=ngerman-x-latest]{hyphsubst}

\usepackage{amsmath, amssymb, amsfonts, amsthm}
\usepackage{mathtools}
\usepackage{array}
\usepackage{stmaryrd}
\usepackage{marvosym}
\usepackage{subcaption}
\allowdisplaybreaks{}
\usepackage[output-decimal-marker={,}]{siunitx}
\usepackage[shortlabels]{enumitem}
\usepackage[section]{placeins}
\usepackage{float}
\usepackage{units}
\usepackage{listings}
\usepackage{pgfplots}
\pgfplotsset{compat=1.12}
\usepackage{tikz}
\usepackage{tikz-qtree}
\usetikzlibrary{arrows,automata}
\usepackage{pdflscape}
\usepackage{xcolor}
\definecolor{light-gray}{HTML}{cccccc}
\usepackage{algorithm}
\usepackage{algpseudocode}
\usepackage{hhline}

\newcommand{\approxalg}[1]{$#1$-Ap\-pro\-xi\-ma\-ti\-ons\-al\-go\-rith\-mus}

\newtheorem*{proposition}{Behauptung}
\newtheorem*{definition}{Definition}
\newcommand{\gdw}{\ \Leftrightarrow\ }
\newcommand{\N}{\mathbb{N}}
\newcommand{\Oh}{\mathcal{O}}
\DeclareMathOperator{\im}{im}
\DeclareMathOperator{\Pr}{Pr}
\DeclareMathOperator{\E}{E}

\lstset{%
    frame=single,
    numbers=left,
    keepspaces,
    language=R,
    title=Listing: \lstname,
}

\def \blattnr {10}

\lohead{VKO -- Blatt {\blattnr}}
\rohead{Alina Bombeck, Lennart Braun, Carolin Konietzny, Tronje Krabbe}
\cofoot{\pagemark}
\setkomafont{pageheadfoot}{\textrm}
\KOMAoption{headsepline}{0.5pt:headsepline}
\pagestyle{scrheadings}


\title{Vertiefung Kombinatorische Optimierung}
\subtitle{Blatt {\blattnr} Hausaufgaben}
\author{%
    Alina Bombeck, 6535392 (Gruppe 1) \and
    Lennart Braun, 6523742 (Gruppe 1) \and
    Carolin Konietzny, 6523939 (Gruppe 1) \and
    Tronje Krabbe, 6435002 (Gruppe 3)
}
\date{zum 4. Juli 2016}
\usepackage{pdfpages}

\begin{document}
\maketitle


\begin{enumerate}[label=\bfseries \arabic*.]
\item % 1.
    Seien $d = \num{80000}$, $r = \num{20000}$.
    $X$ gebe die Anzahl der Stimmen für $D$ an.
    $X_i$ sei $1$, wenn Wähler $i$ für $D$ stimmt, sonst $0$.
    Die ersten $d$ Wähler sind Anhänger von $D$, die folgenden $r$ von $R$.
    Dann ist
    \begin{equation*}
        X = \sum_{i=1}^{d+r} X_i.
    \end{equation*}
    Nach Aufgabenstellung gilt
    \begin{equation*}
        \Pr[X_i = 1] =
        \begin{cases}
            \frac{99}{100}, 1 \leq i \leq d \\
            \frac{1}{100}, d+1 \leq i \leq d+r. \\
        \end{cases}
    \end{equation*}
    Mit der Linearität des Erwartungswertes folgt
    \begin{equation*}
        \begin{gathered}
            \E[X]
            = \E\left[ \sum_{i=1}^{d+r} X_i \right]
            = \sum_{i=1}^{d+r} \E[X_i]
            = \sum_{i=1}^{d+r} \Pr[X_i = 1] \\
            = \sum_{i=1}^{d} \frac{99}{100} + \sum_{i=d+1}^{d+r} \frac{1}{100}
            = d \cdot \frac{99}{100} + r \cdot \frac{1}{100}
            = \num{79400}.
        \end{gathered}
    \end{equation*}


\item % 2.
\begin{enumerate}
    \item
        $S$ ist konfliktfrei, da ein Knoten $v_i$ nur enthalten ist, wenn
        $x_i = 1$ und $x_j = 0$ für alle Nachbarn $v_j$ von $v_i$ gilt; wegen
        letzterer Bedingung können die Nachbarn nicht in $S$ enthalten sein.

        Seien $X_i := \mathbf{1}_S(v_i)$ die Indikatorzufallsvariable, die
        anzeigt, ob $v_i$ in $S$ enthalten ist und $X = \sum_{v_i \in V} X_i$.

        Sei $\Pr[x_i = 1] = p = 1-q$ und $\Pr[x_i = 0] = (1-p) = q$ für alle
        $v_i$.
        Zunächst ist $q = \frac{1}{2}$.

        Da die $x_i$ unabhängig sind, gilt
        \begin{equation*}
            \Pr[X_i = 1]
            = \Pr[x_i = 1] \cdot \prod_{v_j \in N(v_i)} \Pr[x_j = 0]
            = (1-q) \cdot q^d
            = \frac{1}{2^{d+1}}
        \end{equation*}
        Wobei $N(v_i)$ die Nachbarschaft von $v_i$ ist, mit $|N(v_i)| = d$.

        Für den Erwartungswert folgt
        \begin{equation*}
            \E[X]
            = \E\left[ \sum_{v_i \in V} X_i \right]
            = \sum_{v_i \in V} \E[X_i]
            = \sum_{v_i \in V} \Pr[X_i = 1]
            = n \cdot \frac{1}{2^{d+1}}.
        \end{equation*}

    \item
        Nun betrachten wir den Fall, dass $q$ nicht unbedingt gleich
        $\frac{1}{2}$ ist.
        Der Erwartungswert von $X$ ist minimal, wenn der Erwartungswert von den
        $X_i$ minimal ist.
        \begin{equation*}
            \frac{\mathrm{d}}{\mathrm{d}q} \E[X_i]
            = \frac{\mathrm{d}}{\mathrm{d}q} (1-q)q^d
            = \frac{\mathrm{d}}{\mathrm{d}q} q^d - q^{d+1}
            = dq^{d-1} -(d+1)q^d
        \end{equation*}

        Wir bestimmen das Maximum des Erwartungswertes:
        \begin{equation*}
            \begin{aligned}
                & dq^{d-1} -(d+1)q^d = 0 \\
                \gdw& q^{d-1}(d-(d+1)q) = 0 \\
                \gdw& d-(d+1)q = 0 \qquad\qquad\qquad\qquad \text{da } q \neq 0 \\
                \gdw& d = (d+1)q \\
                \gdw& \frac{d}{d+1} = q
            \end{aligned}
        \end{equation*}
        Bei $\frac{d}{d+1}$ ist also ein Extremum vorhanden.
        Es ist
        \begin{equation*}
            \frac{\mathrm{d}^2}{\mathrm{d}q^2} \E[X_i]
            = d(d-1)q^{d-2} - (d+1)dq^{d-1}.
        \end{equation*}
        $\frac{d}{d+1}$ eingesetzt:
        \begin{equation*}
            \begin{gathered}
                d(d-1)\left(\frac{d}{d+1}\right)^{d-2}
                - (d+1)d\left(\frac{d}{d+1}\right)^{d-1} \\
                 = \left(\frac{d}{d+1}\right)^{d-2}
                 \left(d(d-1) - (d+1)d \cdot \frac{d}{d+1}\right) \\
                 = \underbrace{\left(\frac{d}{d+1}\right)^{d-2}}_{> 0}
                 \underbrace{(d(d-1) - d^2)}_{< 0} < 0
            \end{gathered}
        \end{equation*}
        Es gibt damit ein Maximum bei $p = 1-q = \frac{1}{d+1}$.

        Bei optimaler Wahl von $p$ ist dann
        \begin{equation*}
            \E[X] = n \cdot (1-q) \cdot q^d
            = n \cdot \frac{1}{d+1} \cdot \left(\frac{d}{d+1}\right)
            = n \cdot \frac{d^d}{(d+1)^{d+1}}.
        \end{equation*}


\end{enumerate}


\end{enumerate}
\end{document}
