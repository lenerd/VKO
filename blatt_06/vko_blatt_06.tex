\documentclass[a4paper]{scrartcl}

% font/encoding packages
\usepackage[utf8]{inputenc}
\usepackage[T1]{fontenc}
\usepackage{lmodern}
\usepackage[ngerman]{babel}
\usepackage[ngerman=ngerman-x-latest]{hyphsubst}

\usepackage{amsmath, amssymb, amsfonts, amsthm}
\usepackage{mathtools}
\usepackage{array}
\usepackage{stmaryrd}
\usepackage{marvosym}
\usepackage{subcaption}
\allowdisplaybreaks{}
\usepackage[output-decimal-marker={,}]{siunitx}
\usepackage[shortlabels]{enumitem}
\usepackage[section]{placeins}
\usepackage{float}
\usepackage{units}
\usepackage{listings}
\usepackage{pgfplots}
\pgfplotsset{compat=1.12}
\usepackage{tikz}
\usepackage{tikz-qtree}
\usetikzlibrary{arrows,automata}
\usepackage{pdflscape}
\usepackage{xcolor}
\definecolor{light-gray}{HTML}{cccccc}
\usepackage{algorithm}
\usepackage{algpseudocode}

\newcommand{\approxalg}[1]{$#1$-Ap\-pro\-xi\-ma\-ti\-ons\-al\-go\-rith\-mus}

\newtheorem*{proposition}{Behauptung}
\newtheorem*{definition}{Definition}
\newcommand{\gdw}{\ \Leftrightarrow\ }
\newcommand{\N}{\mathbb{N}}
\newcommand{\Oh}{\mathcal{O}}
\DeclareMathOperator{\im}{im}

\usepackage{fancyhdr}
\pagestyle{fancy}

\lstset{%
    frame=single,
    numbers=left,
    keepspaces,
    language=R,
    title=Listing: \lstname,
}

\def \blattnr {6}

\lhead{VKO -- Blatt {\blattnr}}
\rhead{Alina Bombeck, Lennart Braun, Carolin Konietzny, Tronje Krabbe}
\cfoot{\thepage}


\title{Vertiefung Kombinatorische Optimierung}
\subtitle{Blatt {\blattnr} Hausaufgaben}
\author{%
    Alina Bombeck, 6535392 (Gruppe 1) \and
    Lennart Braun, 6523742 (Gruppe 1) \and
    Carolin Konietzny, 6523939 (Gruppe 1) \and
    Tronje Krabbe, 6435002 (Gruppe 3)
}
\date{zum 30. Mai 2016}
\usepackage{pdfpages} 

\begin{document}
\maketitle


\begin{enumerate}[label=\bfseries \arabic*.]
\item % 1.
\begin{enumerate}
    \item % a)
        Wir wollen nach Abschluss des Algorithmus ein möglichst großes $T$ erhalten.
        Wenn $T^*$ das Optimum ist, dann findet ein 2-Approximationsalgorithmus
        ein $T$ mit $2T \geq T^*$.
        Wir geben ein Gegenbeispiel, um zu zeigen, dass der gegebene Algorithmus
        \textit{kein} 2-Approximationsalgorithmus ist:

        Sei $B = 5$ und $A = \{1, 5, 5\}$. Offensichtlich ist $T^* = 5$
        mit $S= \{5\}$, aber der Algorithmus findet $T = 1$ mit $S = \{1\}$.
        \qed
    \item % b)
        Wir ergänzen den Algorithmus aus 1a) um einen Sortieralgorithmus
        mit Laufzeit $\Oh(n \log n)$, z.B. Merge-Sort oder Heap-Sort, und
        sortieren damit $A$ nach absteigender Elementgröße (d.h. das größte
        Element ist $a_1$).
        Danach wenden wir einfach wieder den Algorithmus aus 1a) an, und erhalten
        eine Laufzeit von $\Oh(n \log n + n)$, was in $\Oh(n \log n)$ liegt.
        \\
        \begin{proposition}
            Unser Algorithmus ist ein \approxalg{2}.
        \end{proposition}
        %\begin{proof}
        %    Unser Algorithmus fügt automatisch das größte Element kleiner $B$ in $S$ ein.
        %    Existiert ein solches Element in $S$, das auch größer als $\frac{1}{2}B$
        %    ist, so gilt schon $2T \geq T^*$, denn $T^* \leq B$ gilt immer.
        %    Gibt es kein Element $\geq \frac{1}{2}B$ und $<B$, so fügt unser Algorithmus
        %    zwangsweise genug Elemente hinzu, bis entweder $T \geq \frac{1}{2}B$, oder alle
        %    validen Elemente verbraucht sind. Im ersten Fall ist die Annahme wieder offensichtlich erfüllt.
        %    Im zweiten Fall gilt offensichtlich:
        %    \begin{align*}
        %        T = T^* = \sum_{i=0}^{n}a_i | a_i \in A \wedge a_i < B
        %    \end{align*}
        %    $\square$
        %\end{proof}
        \begin{proof}
            Wir gehen von folgenden Bedingungen aus (beide können in linearer
            Zeit überprüft werden).
            \begin{itemize}
                \item $\forall a_i \in A : a_i \leq B$ (keine unnützen Elemente)
                \item $\sum_{a_i \in A} a_i > B$ (keine triviale Lösung)
            \end{itemize}

            Sei $S^\ast$ eine optimale Lösung mit Summe $T^\ast$ und $S$ die
            Lösung des angegebenen Algorithmus mit Summe $T$.

            Sei $A = \{a_1, a_2, \dotsc, a_n\}$ mit $i < j \gdw a_i < a_j$.
            Da $T^\ast \leq B$ sein muss, folgt $\frac{T^\ast}{2} \leq \frac{B}{2}$.

            Nun unterscheiden wir zwei Fälle:
            \begin{itemize}
                \item $\exists a_i \in A : a_i \geq \frac{B}{2}$: \\
                    Dann ist auch $a_1 \geq \frac{B}{2}$. Da $a_1$ das größte
                    Element ist und $a_1 < B$ gilt, wird $a_1$ auf jeden Fall
                    (und zuerst) gewählt.
                    Damit gilt $T \geq \frac{B}{2} \geq \frac{T^\ast}{2}$
                \item $\lnot\exists a_i \in A : a_i \geq \frac{B}{2}$: \\
                    Angenommen, es sei $T < \frac{B}{2}$. Dann müssen noch
                    ungewählte Elemente existieren: $A \setminus S \neq \emptyset$.
                    Da $\forall a_i \in A : a_i < \frac{B}{2}$ gilt, hätte
                    jedes übrig gebliebene Element gewählt werden können.
                    Dies ist ein Widerspruch zu der im Algorithmus
                    spezifizierten Bedingung: $T + a_i \leq B \Rightarrow a_i$
                    wird gewählt.
                    Daher muss die Annahme falsch und
                    $T \geq \frac{B}{2} \geq \frac{T^\ast}{2}$ sein.
            \end{itemize}
            In beiden Fällen gilt $2 \geq \frac{T^\ast}{T}$; damit handelt es
            sich um einen \approxalg{2}.
        \end{proof}

\end{enumerate}

\item % 2.
    Zu zeigen: $\frac{4}{3}$ ist der bestmögliche, konstante Gütegarantiefaktor.

    Greedy-Balance mit LIF-Regel teilt die in dem Beispiel auf Seite 32 beschriebenen Jobs so
    auf die Maschinen auf, dass alle die gleiche Last haben, mit der Ausnahme von einer Maschine,
    die den dritten Job mit Laufzeit $m$ erhält.\\
    Verteilt man zuerst die ersten $2m$ Jobs, hat jede Maschine eine Last von $m + (2m-1) = 3m-1$, wobei der Job
    der Laufzeit $m$ übrig bleibt; dieser wird dann einer beliebigen Maschine zugeteilt.
    So ergibt sich eine Produktionsspanne $L=4m-1$, während die optimale Produktionsspanne
    $L^*=3m$ beträgt.

    Diese ergibt sich so: Die erste Maschine bekommt die drei Jobs der Länge $m$; alle weiteren Maschinen
    erhalten nur 2 Jobs, nach folgendem Prinzip: jede Maschine, die noch nicht zwei Jobs hat, bekommt
    sowohl den längsten als auch den kürzesten verbleibenden Job.

    Veranschaulicht:
    \begin{itemize}
        \item Maschine ``1'' hat drei Jobs mit der Laufzeit $m$.
        \item Maschine ``2'' hat zwei Jobs mit den Laufzeiten $m+1$ und $2m-1$.
        \item Maschine ``3'' wie Maschine 2.
        \item Maschine ``4'' hat zwei Jobs mit den Laufzeiten $m+2$ und $2m-2$.
        \item usw.
    \end{itemize}
    So ergibt sich für jede Maschine eine Laufzeit von $L^*=3m$.

    Es gilt
    \begin{equation*}
        L = 4m - 1 < 4m = \frac{4}{3} \cdot 3m = \frac{4}{3} L^\ast
    \end{equation*}

    Gäbe es einen besseren konstanten Gütegarantiefaktor $x$, müsste er $4m \cdot x = 4m-1$ erfüllen.
    Daraus würde sich ergeben:
    $x = \frac{m-1}{m}$, was nicht konstant ist.
    \qed
\end{enumerate}
\end{document}
