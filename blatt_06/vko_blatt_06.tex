\documentclass[a4paper]{scrartcl}

% font/encoding packages
\usepackage[utf8]{inputenc}
\usepackage[T1]{fontenc}
\usepackage{lmodern}
\usepackage[ngerman]{babel}
\usepackage[ngerman=ngerman-x-latest]{hyphsubst}

\usepackage{amsmath, amssymb, amsfonts, amsthm}
\usepackage{mathtools}
\usepackage{array}
\usepackage{stmaryrd}
\usepackage{marvosym}
\usepackage{subcaption}
\allowdisplaybreaks{}
\usepackage[output-decimal-marker={,}]{siunitx}
\usepackage[shortlabels]{enumitem}
\usepackage[section]{placeins}
\usepackage{float}
\usepackage{units}
\usepackage{listings}
\usepackage{pgfplots}
\pgfplotsset{compat=1.12}
\usepackage{tikz}
\usepackage{tikz-qtree}
\usetikzlibrary{arrows,automata}
\usepackage{pdflscape}
\usepackage{xcolor}
\definecolor{light-gray}{HTML}{cccccc}
\usepackage{algorithm}
\usepackage{algpseudocode}


\newtheorem*{proposition}{Behauptung}
\newtheorem*{definition}{Definition}
\newcommand{\gdw}{\ \Leftrightarrow\ }
\newcommand{\N}{\mathbb{N}}
\newcommand{\Oh}{\mathcal{O}}
\DeclareMathOperator{\im}{im}

\usepackage{fancyhdr}
\pagestyle{fancy}

\lstset{%
    frame=single,
    numbers=left,
    keepspaces,
    language=R,
    title=Listing: \lstname,
}

\def \blattnr {6}

\lhead{VKO -- Blatt {\blattnr}}
\rhead{Alina Bombeck, Lennart Braun, Carolin Konietzny, Tronje Krabbe}
\cfoot{\thepage}


\title{Vertiefung Kombinatorische Optimierung}
\subtitle{Blatt {\blattnr} Hausaufgaben}
\author{%
    Alina Bombeck, 6535392 (Gruppe 1) \and
    Lennart Braun, 6523742 (Gruppe 1) \and
    Carolin Konietzny, 6523939 (Gruppe 1) \and
    Tronje Krabbe, 6435002 (Gruppe 3)
}
\date{zum 30. Mai 2016}
\usepackage{pdfpages} 

\begin{document}
\maketitle


\begin{enumerate}[label=\bfseries \arabic*.]
\item % 1.
\begin{enumerate}
    \item % a)
        Wir wollen nach Abschluss des Algorithmus ein möglichst großes $T$ erhalten.
        Wenn $T^*$ das Optimum ist, dann findet ein 2-Approximationsalgorithmus
        ein $T$ mit $2T \leq T^*$.
        Wir geben ein Gegenbeispiel, um zu zeigen, dass der gegebene Algorithmus
        \textit{kein} 2-Approximationsalgorithmus ist:\\
        Sei $B = 5$ und $S = \{1, 5, 5\}$. Offensichtlich ist $T^* = 5$
        mit $S= \{5\}$, aber der Algorithmus findet $T = 1$ mit $S = \{1\}$.
        $\square$
    \item % b)
        TODO

\end{enumerate}

\item % 2.
\begin{enumerate}
    \item % a)
        TODO
    \item % b)
        TODO
\end{enumerate}

\end{enumerate}
\end{document}
