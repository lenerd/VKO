\documentclass[a4paper]{scrartcl}

% font/encoding packages
\usepackage[utf8]{inputenc}
\usepackage[T1]{fontenc}
\usepackage{lmodern}
\usepackage[ngerman]{babel}
\usepackage[ngerman=ngerman-x-latest]{hyphsubst}

\usepackage{amsmath, amssymb, amsfonts, amsthm}
\usepackage{mathtools}
\usepackage{array}
\usepackage{stmaryrd}
\usepackage{marvosym}
\usepackage{subcaption}
\allowdisplaybreaks{}
\usepackage[output-decimal-marker={,}]{siunitx}
\usepackage[shortlabels]{enumitem}
\usepackage[section]{placeins}
\usepackage{float}
\usepackage{units}
\usepackage{listings}
\usepackage{pgfplots}
\pgfplotsset{compat=1.12}
\usepackage{tikz}
\usepackage{tikz-qtree}
\usetikzlibrary{arrows,automata}
\usepackage{pdflscape}
\usepackage{xcolor}
\definecolor{light-gray}{HTML}{cccccc}
\usepackage{algorithm}
\usepackage{algpseudocode}


\newtheorem*{proposition}{Behauptung}
\newtheorem*{definition}{Definition}
\newcommand{\gdw}{\ \Leftrightarrow\ }
\newcommand{\N}{\mathbb{N}}
\newcommand{\Oh}{\mathcal{O}}
\DeclareMathOperator{\im}{im}

\usepackage{fancyhdr}
\pagestyle{fancy}

\lstset{%
    frame=single,
    numbers=left,
    keepspaces,
    language=R,
    title=Listing: \lstname,
}

\def \blattnr {6}

\lhead{VKO -- Blatt {\blattnr}}
\rhead{Alina Bombeck, Lennart Braun, Carolin Konietzny, Tronje Krabbe}
\cfoot{\thepage}


\title{Vertiefung Kombinatorische Optimierung}
\subtitle{Blatt {\blattnr} Hausaufgaben}
\author{%
    Alina Bombeck, 6535392 (Gruppe 1) \and
    Lennart Braun, 6523742 (Gruppe 1) \and
    Carolin Konietzny, 6523939 (Gruppe 1) \and
    Tronje Krabbe, 6435002 (Gruppe 3)
}
\date{zum 30. Mai 2016}
\usepackage{pdfpages} 

\begin{document}
\maketitle


\begin{enumerate}[label=\bfseries \arabic*.]
\item % 1.
\begin{enumerate}
    \item % a)
        Wir wollen nach Abschluss des Algorithmus ein möglichst großes $T$ erhalten.
        Wenn $T^*$ das Optimum ist, dann findet ein 2-Approximationsalgorithmus
        ein $T$ mit $2T \geq T^*$.
        Wir geben ein Gegenbeispiel, um zu zeigen, dass der gegebene Algorithmus
        \textit{kein} 2-Approximationsalgorithmus ist:\\
        Sei $B = 5$ und $A = \{1, 5, 5\}$. Offensichtlich ist $T^* = 5$
        mit $S= \{5\}$, aber der Algorithmus findet $T = 1$ mit $S = \{1\}$.
        $\square$
    \item % b)
        Wir ergänzen den Algorithmus aus 1a) um einen Sortieralgorithmus
        mit Laufzeit $O(n \log n)$, z.B. Merge-Sort oder Heap-Sort, und
        sortieren damit $S$ nach absteigender Elementgröße (d.h. das größte
        Element ist $a_1$).
        Danach wenden wir einfach wieder den Algorithmus aus 1a) an, und erhalten
        eine Laufzeit von $O(n \log n + n)$, was in $O(n \log n)$ liegt.
        \\
        \begin{proposition}
        Unser Algorithmus ist ein 2-Approximationsalgorithmus.
        \end{proposition}
        \begin{proof}
            Unser Algorithmus fügt automatisch das größte Element kleiner $B$ in $S$ ein.
            Existiert ein solches Element in $S$, das auch größer als $\frac{1}{2}B$
            ist, so gilt schon $2T \geq T^*$, denn $T^* \leq B$ gilt immer.
            Gibt es kein Element $\geq \frac{1}{2}B$ und $<B$, so fügt unser Algorithmus
            zwangsweise genug Elemente hinzu, bis entweder $T \geq \frac{1}{2}B$, oder alle
            validen Elemente verbraucht sind. Im ersten Fall ist die Annahme wieder offensichtlich erfüllt.
            Im zweiten Fall gilt offensichtlich:
            \begin{align*}
                T = T^* = \sum_{i=0}^{n}a_i | a_i \in A \wedge a_i < B
            \end{align*}
            $\square$
        \end{proof}

\end{enumerate}

\item % 2.
    Zu zeigen: $\frac{4}{3}$ ist der bestmögliche, konstante Gütegarantiefaktor.\\
    Greedy-Balance mit LIF-Regel teilt die in dem Beispiel auf Seite 32 beschriebenen Jobs so
    auf die Maschinen auf, dass alle die gleiche Last haben, mit der Ausnahme von einer Maschine,
    die den dritten Job mit Laufzeit $m$ erhält.\\
    Verteilt man zuerst die ersten $2m$ Jobs, hat jede Maschine eine Last von $m + (2m-1) = 3m-1$, wobei der Job
    der Laufzeit $m$ übrig bleibt; dieser wird dann einer beliebigen Maschine zugeteilt.
    So ergibt sich eine Produktionsspanne $L=4m-1$, während die optimale Produktionsspanne
    $L^*=3m$ beträgt.
    \\
    Diese ergibt sich so: die erste Maschine bekommt die drei Jobs der Länge $m$; alle weiteren Maschinen
    erhalten nur 2 Jobs, nach folgendem Prinzip: jede Maschine, die noch nicht zwei Jobs hat, bekommt
    sowohl den längsten als auch den kürzesten verbleibenden Job.\\
    \newpage
    Veranschauligt:\\
    Maschine ``1'' hat drei Jobs mit der Laufzeit $m$.\\
    Maschine ``2'' hat zwei Jobs mit den Laufzeiten $m+1$ und $2m-1$.\\
    Maschine ``3'' wie Maschine 2.\\
    Maschine ``4'' hat zwei Jobs mit den Laufzeiten $m+2$ und $2m-2$.\\
    usw. So ergibt sich für jede Maschine eine Laufzeit von $L^*=3m$.
    \\ \\
    $L > \frac{4}{3}L^*$ ($L$ ist die vom Algorithmus gelieferte Produktionsspanne, und $L^*$ ist die
    optimale Produktionsspanne).\\
    Außerdem:\\
    $4m-1 \leq \frac{4}{3} 3m = 4m$ und\\
    $4m-1 \leq 4m$.

    Gäbe es einen besseren konstanten Gütegarantiefaktor $x$, müsste er $4m \cdot x = 4m-1$ erfüllen.
    Daraus würde sich ergeben:
    $x = \frac{m-1}{m}$, was nicht konstant ist. $\square$
\end{enumerate}
\end{document}
