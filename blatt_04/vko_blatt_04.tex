\documentclass[a4paper]{scrartcl}

% font/encoding packages
\usepackage[utf8]{inputenc}
\usepackage[T1]{fontenc}
\usepackage{lmodern}
\usepackage[ngerman]{babel}
\usepackage[ngerman=ngerman-x-latest]{hyphsubst}

\usepackage{amsmath, amssymb, amsfonts, amsthm}
\usepackage{mathtools}
\usepackage{array}
\usepackage{stmaryrd}
\usepackage{marvosym}
\usepackage{subcaption}
\allowdisplaybreaks{}
\usepackage[output-decimal-marker={,}]{siunitx}
\usepackage[shortlabels]{enumitem}
\usepackage[section]{placeins}
\usepackage{float}
\usepackage{units}
\usepackage{listings}
\usepackage{pgfplots}
\pgfplotsset{compat=1.12}
\usepackage{tikz}
\usepackage{tikz-qtree}
\usetikzlibrary{arrows,automata}
\usepackage{pdflscape}
\usepackage{xcolor}
\definecolor{light-gray}{HTML}{cccccc}


\newtheorem*{proposition}{Behauptung}
\newtheorem*{definition}{Definition}
\newcommand{\gdw}{\ \Leftrightarrow\ }
\newcommand{\N}{\mathbb{N}}
\newcommand{\Oh}{\mathcal{O}}
\DeclareMathOperator{\im}{im}

\usepackage{fancyhdr}
\pagestyle{fancy}

\lstset{%
    frame=single,
    numbers=left,
    keepspaces,
    language=R,
    title=Listing: \lstname,
}

\def \blattnr {4}

\lhead{VKO -- Blatt {\blattnr}}
\rhead{Alina Bombeck, Lennart Braun, Carolin Konietzny, Tronje Krabbe}
\cfoot{\thepage}


\title{Vertiefung Kombinatorische Optimierung}
\subtitle{Blatt {\blattnr} Hausaufgaben}
\author{%
    Alina Bombeck, 6535392 (Gruppe 1) \and
    Lennart Braun, 6523742 (Gruppe 1) \and
    Carolin Konietzny, 6523939 (Gruppe 1) \and
    Tronje Krabbe, 6435002 (Gruppe 3)
}
\date{zum 9. Mai 2016}
\usepackage{pdfpages} 

\begin{document}
\maketitle


\begin{enumerate}[label=\bfseries \arabic*.]
\item % 1.
\begin{enumerate}
    \item % a)
        Baum siehe Abbildung \ref{fig:baum1a}
    \item % b)
        Baum siehe Abbildung \ref{fig:baum1b}
\end{enumerate}

\item % 2.
    \begin{figure}
        \begin{tikzpicture}[every tree node/.style={align=center,anchor=north,rectangle,draw}, level distance=70pt,scale=0.8]
\Tree [. $F(x)$
        % 1st edge to the left
        \edge node[auto=right] {$e_{34}$};
        [.$S(e_{34})$\\23 ]
        % 1st edge to the right
        \edge node[auto=left] {$\bar{e}_{34}$};
        [.$S(\bar{e}_{34})$
            % left
            \edge node[auto=right] {$e_{45}$};
            [.$S(\bar{e}_{34},e_{45})$\\20 ]
            % right
            \edge node[auto=left] {$\bar{e}_{45}$};
            [.$S(\bar{e}_{34},\bar{e}_{45})$
                % left
                \edge node[auto=right] {$e_{13}$};
                [.$S(e_{13},\bar{e}_{34},\bar{e}_{45})$\\19 ]
                % right
                \edge node[auto=left] {$\bar{e}_{13}$};
                [.$S(\bar{e}_{13},\bar{e}_{34},\bar{e}_{45})$
                    % left
                    \edge node[auto=right] {$e_{25}$};
                    [.{$S(\bar{e}_{13},e_{25},\bar{e}_{34},\bar{e}_{45})$\\
                        Kein Ham.-Kreis möglich!} ]
                    % right
                    \edge node[auto=left] {$\bar{e}_{25}$};
                    [.$S(\bar{e}_{13},\bar{e}_{25},\bar{e}_{34},\bar{e}_{45})$
                        % left
                        \edge node[auto=right] {$e_{12}$};
                        [.{$S(e_{12},\bar{e}_{13},\bar{e}_{25},\bar{e}_{34},\bar{e}_{45})$
                        \\ Kein Ham.-Kreis möglich!} ]
                        % right
                        \edge node[auto=left] {$\bar{e}_{12}$};
                        [.$S(\bar{e}_{12},\bar{e}_{13},\bar{e}_{25},\bar{e}_{34},\bar{e}_{45})$\\17 ]
                    ]
                ]
            ]
        ]
    ]
\end{tikzpicture}

        \caption{}
    \end{figure}

\end{enumerate}

\begin{landscape}
    \begin{figure}
        \centering
        \begin{tikzpicture}[%
    scale=2,
    vertex/.style = {draw,circle,},
    level 1/.style={sibling distance=5cm},
    level 2/.style={sibling distance=2.0cm},
    level 3/.style={sibling distance=1cm},
    level 4/.style={sibling distance=0.5cm},
]
\node (root) [vertex] {}
    child {
        node[vertex] {}
        child {
            node[vertex] {}
            child {
                node[vertex] {}
                child {
                    node[vertex] {}
                    node [yshift=-1cm] {9}
                    edge from parent node[left] {$x_2 = 1$} 
                }
                child {
                    node[vertex] {}
                    node [yshift=-1cm] {8}
                    edge from parent node[right] {$0$} 
                }
                edge from parent node[left] {$x_1 = 1$} 
            }
            child {
                node[vertex] {}
                child {
                    node[vertex] {}
                    node [yshift=-1cm] {7}
                    edge from parent node[left] {$1$} 
                }
                child {
                    node[vertex] {}
                    node [yshift=-1cm] {8}
                    edge from parent node[right] {$0$} 
                }
                edge from parent node[right] {$x_1 = 0$} 
            }
            edge from parent node[left] {$x_3 = 1$} 
        }
        child {
            node[vertex] {}
            child {
                node[vertex] {}
                child {
                    node[vertex] {}
                    node [yshift=-1cm] {9}
                    edge from parent node[left] {$1$} 
                }
                child {
                    node[vertex] {}
                    node [yshift=-1cm] {7}
                    edge from parent node[right] {$0$} 
                }
                edge from parent node[left] {$x_1 = 1$} 
            }
            child {
                node[vertex] {}
                child {
                    node[vertex] {}
                    node [yshift=-1cm] {6}
                    edge from parent node[left] {$1$} 
                }
                child {
                    node[vertex] {}
                    node [yshift=-1cm] {6}
                    edge from parent node[right] {$0$} 
                }
                edge from parent node[right] {$x_1 = 0$} 
            }
            edge from parent node[right] {$x_3 = 0$} 
        }
        edge from parent node[left] {$x_4 = 1$} 
    }
    child {
        node[vertex] {}
        child {
            node[vertex] {}
            child {
                node[vertex] {}
                child {
                    node[vertex] {}
                    node [yshift=-1cm] {9}
                    edge from parent node[left] {$1$} 
                }
                child {
                    node[vertex] {}
                    node [yshift=-1cm] {8}
                    edge from parent node[right] {$0$} 
                }
                edge from parent node[left] {$x_1 = 1$} 
            }
            child {
                node[vertex] {}
                child {
                    node[vertex] {}
                    node [yshift=-1cm] {6}
                    edge from parent node[left] {$1$} 
                }
                child {
                    node[vertex] {}
                    node [yshift=-1cm] {7}
                    edge from parent node[right] {$0$} 
                }
                edge from parent node[right] {$x_1 = 0$} 
            }
            edge from parent node[left] {$x_3 = 1$} 
        }
        child {
            node[vertex] {}
            child {
                node[vertex] {}
                child {
                    node[vertex] {}
                    node [yshift=-1cm] {9}
                    edge from parent node[left] {$1$} 
                }
                child {
                    node[vertex] {}
                    node [yshift=-1cm] {8}
                    edge from parent node[right] {$0$} 
                }
                edge from parent node[left] {$x_1 = 1$} 
            }
            child {
                node[vertex] {}
                child {
                    node[vertex] {}
                    node [yshift=-1cm] {6}
                    edge from parent node[left] {$1$} 
                }
                child {
                    node[vertex] {}
                    node [yshift=-1cm] {7}
                    edge from parent node[right] {$0$} 
                }
                edge from parent node[right] {$x_1 = 0$} 
            }
            edge from parent node[right] {$x_3 = 0$} 
        }
        edge from parent node[right] {$x_4 = 0$} 
    }
    ;

\end{tikzpicture}

        \caption{Backtrack-Baum $T_{\mathcal{F}(x)}$}
        \label{fig:baum1a}
    \end{figure}
    \begin{figure}
        \centering
        
\begin{tikzpicture}[%
    scale=2,
    vertex/.style = {draw,circle,},
    level 1/.style={sibling distance=5cm},
    level 2/.style={sibling distance=2.0cm},
    level 3/.style={sibling distance=1cm},
    level 4/.style={sibling distance=0.5cm},
]
\node (root) [vertex] {}
    child {
        node[vertex] {}
        child {
            node[vertex] {}
            child {
                node[vertex] {}
                child {
                    node[vertex] {}
                    node[yshift=-1cm] {9}
                    edge from parent node[left] {$x_2 = 1$} 
                }
                child [edge from parent/.style={draw=none}] {}
                node[below right]
                {$\{\overline{x_1} \lor \overline{x_3}\}$}
                edge from parent node[left] {$x_1 = 1$} 
            }
            child {
                node[vertex] {}
                node[below right]
                {$\{x_1\}$}
                edge from parent node[right] {$x_1 = 0$} 
            }
            edge from parent node[left] {$x_3 = 1$} 
        }
        child {
            node[vertex] {}
            node[yshift=-0.5cm] {$\{x_3\}$}
            edge from parent node[right] {$x_3 = 0$} 
        }
        edge from parent node[left] {$x_4 = 1$} 
    }
    child {
        node[vertex] {}
        child {
            node[vertex] {}
            child {
                node[vertex] {}
                node[yshift=-0.5cm] {$\{\overline{x_1} \lor \overline{x_3}\}$}
                edge from parent node[left] {$x_1 = 1$} 
            }
            child {
                node[vertex] {}
                node[yshift=-0.5cm] {$\{x_1\}$}
                edge from parent node[right] {$x_1 = 0$} 
            }
            edge from parent node[left] {$x_3 = 1$} 
        }
        child {
            node[vertex] {}
            node[yshift=-0.5cm] {$\{x_3\}$}
            edge from parent node[right] {$x_3 = 0$} 
        }
        edge from parent node[right] {$x_4 = 0$} 
    }
    ;

\end{tikzpicture}

        \caption{Branch-and-Bound-Baum}
        \label{fig:baum1b}
    \end{figure}
\end{landscape}


\end{document}

