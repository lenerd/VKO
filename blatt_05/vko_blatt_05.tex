\documentclass[a4paper]{scrartcl}

% font/encoding packages
\usepackage[utf8]{inputenc}
\usepackage[T1]{fontenc}
\usepackage{lmodern}
\usepackage[ngerman]{babel}
\usepackage[ngerman=ngerman-x-latest]{hyphsubst}

\usepackage{amsmath, amssymb, amsfonts, amsthm}
\usepackage{mathtools}
\usepackage{array}
\usepackage{stmaryrd}
\usepackage{marvosym}
\usepackage{subcaption}
\allowdisplaybreaks{}
\usepackage[output-decimal-marker={,}]{siunitx}
\usepackage[shortlabels]{enumitem}
\usepackage[section]{placeins}
\usepackage{float}
\usepackage{units}
\usepackage{listings}
\usepackage{pgfplots}
\pgfplotsset{compat=1.12}
\usepackage{tikz}
\usepackage{tikz-qtree}
\usetikzlibrary{arrows,automata}
\usepackage{pdflscape}
\usepackage{xcolor}
\definecolor{light-gray}{HTML}{cccccc}


\newtheorem*{proposition}{Behauptung}
\newtheorem*{definition}{Definition}
\newcommand{\gdw}{\ \Leftrightarrow\ }
\newcommand{\N}{\mathbb{N}}
\newcommand{\Oh}{\mathcal{O}}
\DeclareMathOperator{\im}{im}

\usepackage{fancyhdr}
\pagestyle{fancy}

\lstset{%
    frame=single,
    numbers=left,
    keepspaces,
    language=R,
    title=Listing: \lstname,
}

\def \blattnr {5}

\lhead{VKO -- Blatt {\blattnr}}
\rhead{Alina Bombeck, Lennart Braun, Carolin Konietzny, Tronje Krabbe}
\cfoot{\thepage}


\title{Vertiefung Kombinatorische Optimierung}
\subtitle{Blatt {\blattnr} Hausaufgaben}
\author{%
    Alina Bombeck, 6535392 (Gruppe 1) \and
    Lennart Braun, 6523742 (Gruppe 1) \and
    Carolin Konietzny, 6523939 (Gruppe 1) \and
    Tronje Krabbe, 6435002 (Gruppe 3)
}
\date{zum 23. Mai 2016}
\usepackage{pdfpages} 

\begin{document}
\maketitle


\begin{enumerate}[label=\bfseries \arabic*.]
\item % 1.
\begin{enumerate}
    \item % a)
        \hfill \\
        \begin{figure}[H]
        \centering
        \begin{tikzpicture}
            \Tree
            [.$t_1$
                [.$t_2$
                    [.$t_3$ [.$t_6$ ][.$t_4$ [.$t_5$ [.$t_9$ ] [.$t_8$ ] ]
                            [.$t_7$ ] ] ] ] ]
        \end{tikzpicture}
        \caption{Baumzerlegung 1a)}
        \end{figure}

        \begin{equation}
        \begin{aligned}
            V_1 &= \{v_3, v_4, v_5 \} \\
            V_2 &= \{v_2, v_3, v_5 \} \\
            V_3 &= \{v_2, v_5, v_7\} \\
            V_4 &= \{v_2, v_7, v_{11}\} \\
            V_5 &= \{v_7, v_9, v_{11}\}
        \end{aligned}
        \quad
        \begin{aligned}
            V_6 &= \{v_5, v_6, v_7\} \\
            V_7 &= \{v_1, v_2, v_{11}\} \\
            V_8 &= \{v_9, v_{10}, v_{11}\} \\
            V_9 &= \{v_7, v_8, v_9\} \\
            \phantom{irgendwas}
        \end{aligned}
        \end{equation}
        
    \item % b)
        TODO
\end{enumerate}

\item % 2.
\begin{enumerate}
    \item % a)
        TODO
    \item % b)
        TODO
\end{enumerate}

\end{enumerate}


\end{document}

