\documentclass[a4paper]{scrartcl}

% font/encoding packages
\usepackage[utf8]{inputenc}
\usepackage[T1]{fontenc}
\usepackage{lmodern}
\usepackage[ngerman]{babel}
\usepackage[ngerman=ngerman-x-latest]{hyphsubst}

\usepackage{amsmath, amssymb, amsfonts, amsthm}
\usepackage{mathtools}
\usepackage{array}
\usepackage{stmaryrd}
\usepackage{marvosym}
\usepackage{subcaption}
\allowdisplaybreaks{}
\usepackage[output-decimal-marker={,}]{siunitx}
\usepackage[shortlabels]{enumitem}
\usepackage[section]{placeins}
\usepackage{float}
\usepackage{units}
\usepackage{listings}
\usepackage{pgfplots}
\pgfplotsset{compat=1.12}
\usepackage{tikz}
\usepackage{tikz-qtree}
\usetikzlibrary{arrows,automata}
\usepackage{pdflscape}
\usepackage{xcolor}
\definecolor{light-gray}{HTML}{cccccc}


\newtheorem*{proposition}{Behauptung}
\newtheorem*{definition}{Definition}
\newcommand{\gdw}{\ \Leftrightarrow\ }
\newcommand{\N}{\mathbb{N}}
\newcommand{\Oh}{\mathcal{O}}
\DeclareMathOperator{\im}{im}

\usepackage{fancyhdr}
\pagestyle{fancy}

\lstset{%
    frame=single,
    numbers=left,
    keepspaces,
    language=R,
    title=Listing: \lstname,
}

\def \blattnr {5}

\lhead{VKO -- Blatt {\blattnr}}
\rhead{Alina Bombeck, Lennart Braun, Carolin Konietzny, Tronje Krabbe}
\cfoot{\thepage}


\title{Vertiefung Kombinatorische Optimierung}
\subtitle{Blatt {\blattnr} Hausaufgaben}
\author{%
    Alina Bombeck, 6535392 (Gruppe 1) \and
    Lennart Braun, 6523742 (Gruppe 1) \and
    Carolin Konietzny, 6523939 (Gruppe 1) \and
    Tronje Krabbe, 6435002 (Gruppe 3)
}
\date{zum 23. Mai 2016}
\usepackage{pdfpages} 

\begin{document}
\maketitle


\begin{enumerate}[label=\bfseries \arabic*.]
\item % 1.
\begin{enumerate}
    \item % a)
        \hfill \\
        \begin{figure}[H]
        \centering
        \begin{tikzpicture}
            \Tree
            [.$t_1$
                [.$t_2$
                    [.$t_3$ [.$t_6$ ][.$t_4$ [.$t_5$ [.$t_9$ ] [.$t_8$ ] ]
                            [.$t_7$ ] ] ] ] ]
        \end{tikzpicture}
        \caption{Baumzerlegung 1a)}
        \end{figure}

        \begin{equation*}
            \begin{aligned}
                V_1 &= \{v_3, v_4, v_5 \} \\
                V_2 &= \{v_2, v_3, v_5 \} \\
                V_3 &= \{v_2, v_5, v_7\} \\
                V_4 &= \{v_2, v_7, v_{11}\} \\
                V_5 &= \{v_7, v_9, v_{11}\}
            \end{aligned}
            \quad
            \begin{aligned}
                V_6 &= \{v_5, v_6, v_7\} \\
                V_7 &= \{v_1, v_2, v_{11}\} \\
                V_8 &= \{v_9, v_{10}, v_{11}\} \\
                V_9 &= \{v_7, v_8, v_9\} \\
                \phantom{irgendwas}
            \end{aligned}
        \end{equation*}
        
    \item % b)
        TODO
\end{enumerate}

\item % 2.
    Im Folgenden bezeichnen wir die Distanzfunktion mit $\hat{d}$.
\begin{enumerate}
    \item % a)
        \begin{proposition}
            Der Aufruf $\textsc{Explore}(\Phi, d)$ gibt genau dann „yes“
            zurück, wenn eine erfüllende Belegung $\Phi'$ mit
            $\hat{d}(\Phi, \Phi') < d$ existiert.

            $\textsc{Explore}(\Phi, d)$ benötigt $\Oh(n^2 \cdot 3^d)$ Zeit.
        \end{proposition}
        \begin{proof}
            \leavevmode
            \begin{itemize}
                \item $\Rightarrow$: \\
                    Der Algorithmus hat „yes“ ausgegeben.  Dann gibt es eine
                    erfüllende Belegung $\Phi'$ und eine Folge von maximal $d$
                    Änderungen von Belegungen jeweils eines Atoms, die $\Phi$
                    in ein $\Phi'$ überführen. Also ist $\hat{d}(\Phi, \Phi')
                    \leq d$.
                \item $\Leftarrow$: \\
                    Sei $\hat{d}(\Phi, \Phi') \leq d$, dann gibt es eine Folge
                    von maximal $d$ Änderungen an der Belegung jeweils eines
                    Atoms, die $\Phi$ in $\Phi'$ überführen.
                    Eine solche Folge wird während der rekursiven Ausführung
                    von \textsc{Explore} erreicht.
                    Also gibt der Algorithmus „yes“ aus.
            \end{itemize}

            Die Rekursionstiefe beträgt offensichtlich $d$, der
            Verzweigungsgrad beträgt $3$.
            Damit ist ein Faktor von $3^d$ in der Laufzeit enthalten.
            In jedem Rekursionsschritt müssen eine von $\Phi$ nicht erfüllt
            Klausel gewählt und die Belegungen $\Phi_i$, $i \in \{1,2,3\}$
            erstellt werden.
            Es gibt maximal $\binom{n}{3} \cdot 2^3 \in \Oh(n^2)$ Klauseln, die
            überprüft werden müssen.
            Daher läuft $\textsc{Explore}(\Phi, d)$ in $\Oh(n^2 \cdot 3^d)$.
        \end{proof}

    \item % b)
        Seien $\Phi_1$ und $\Phi_2$ Belegunden, die allen Atomen den
        Wahrheitswert „wahr“ bzw. „falsch“ zuweist.
        Sei $\Phi'$ eine erfüllende Belegung, dann gilt $\hat{d}(\Phi_1, \Phi')
        \leq \frac{n}{2}$ oder $\hat{d}(\Phi_2, \Phi') \leq \frac{n}{2}$, denn
        es können nicht gleichzeitig mehr als die Hälfte der Atome den
        Wahrheitswert „wahr“ und „falsch“ zugewiesen haben.

        Man führt nun nacheinander
        $\textsc{Explore}(\Phi_1, \lfloor \frac{n}{2} \rfloor)$ und
        $\textsc{Explore}(\Phi_2, \lfloor \frac{n}{2} \rfloor)$ aus und gibt
        „yes“ aus, sobald eine der beiden Aufrufe „yes“ zurückgibt.
        Die Laufzeit beträgt also
        $\Oh \left(2 \cdot (n^2 \cdot 3^{\frac{n}{2}}) \right) = \Oh \left( n^2 \cdot {(\sqrt{3})}^n \right)$.

\end{enumerate}

\end{enumerate}


\end{document}

