\documentclass[a4paper]{scrartcl}

% font/encoding packages
\usepackage[utf8]{inputenc}
\usepackage[T1]{fontenc}
\usepackage{lmodern}
\usepackage[ngerman]{babel}
\usepackage[ngerman=ngerman-x-latest]{hyphsubst}

\usepackage{amsmath, amssymb, amsfonts, amsthm}
\usepackage{mathtools}
\usepackage{array}
\usepackage{stmaryrd}
\usepackage{marvosym}
\usepackage{subcaption}
\allowdisplaybreaks{}
\usepackage[output-decimal-marker={,}]{siunitx}
\usepackage[shortlabels]{enumitem}
\usepackage[section]{placeins}
\usepackage{float}
\usepackage{units}
\usepackage{listings}
\usepackage{pgfplots}
\pgfplotsset{compat=1.12}
\usepackage{tikz}
\usetikzlibrary{arrows,automata}

\usepackage{xcolor}
\definecolor{light-gray}{HTML}{cccccc}


\newtheorem*{proposition}{Behauptung}
\newtheorem*{definition}{Definition}
\newcommand{\gdw}{\ \Leftrightarrow\ }
\newcommand{\N}{\mathbb{N}}
\newcommand{\Oh}{\mathcal{O}}
\DeclareMathOperator{\im}{im}

\usepackage{fancyhdr}
\pagestyle{fancy}

\lstset{%
    frame=single,
    numbers=left,
    keepspaces,
    language=R,
    title=Listing: \lstname,
}

\def \blattnr {1}

\lhead{VKO -- Blatt {\blattnr}}
\rhead{Alina Bombeck, Lennart Braun, Carolin Konietzny, Tronje Krabbe}
\cfoot{\thepage}


\title{Vertiefung Kombinatorische Optimierung}
\subtitle{Blatt {\blattnr} Hausaufgaben}
\author{%
    Alina Bombeck,
    Lennart Braun,
    Carolin Konietzny
    Tronje Krabbe
}
\date{zum 18. April 2016}
\usepackage{pdfpages} 

\begin{document}
\maketitle


\begin{enumerate}[label=\bfseries \arabic*.]
\item % 1.
\begin{enumerate}
\item
    \begin{definition}[\textsc{4D-Matching}]
        Seien $W,X,Y,Z$ Mengen mit je $n$ Elementen.
        Sei $T \subset X \times X \times Y \times Z$.
        Gibt es eine Menge $T' \subset T$ mit $|T'| = n$, so dass jedes Element
        in $W \cup X \cup Y \cup Z$ in genau einem der Tupel in $T'$ vorkommt?
    \end{definition}
    \begin{proposition}
        \textsc{4D-Matching} ist $NP$-vollständig.
    \end{proposition}
    \begin{proof}
        \hfill \\
        \begin{enumerate}
            \item \textsc{4D-Matching} $\in NP$: \\
                Ein Verifikationsalgorithmus kann in polynomieller Zeit zeigen,
                dass eine Ja-Eingabe vorliegt: Das Zertifikat ist dabei die
                Menge der Tupel $T'$. In $n$ Schritten kann über $T'$ iteriert
                werden, wobei in jedem Schritt jeweils genau ein Element aus
                den Mengen $W,X,Y,Z$ entfernt wird. Ist dies nicht möglich,
                handelt es sich nicht um ein gültiges Zertifikat; andernfalls
                wird die Korrektheit der Lösung bestätigt.

            \item \textsc{4D-Matching} ist $NP$-schwierig: \\
                Wir reduzieren \textsc{3D-Matching} auf \textsc{4D-Matching}:
                Sei $\langle X,Y,Z,T \rangle$ eine Eingabe für
                \textsc{3D-Matching} und $n = |X|$.  Wir erzeugen eine weitere
                Menge $W = \{w_1, \dotsc, w_n\}$ und eine Menge an Tupeln
                $\overline{T} = W \times T$.  $\langle W,X,Y,Z,\overline{T}
                \rangle$ ist nun eine Eingabe für \textsc{4D-Matching}.
                Dies ist effizient möglich, da $T$ maximal $n^3$ Elemente und
                $\overline{T}$ maximal $n^4$ Elemente enthalten.

                Wir zeigen nun, dass für $\langle X,Y,Z,T \rangle$ genau dann
                eine Ja-Eingabe für \textsc{3D-Matching} ist, wenn ein
                \textsc{4D-Matching} für $\langle W,X,Y,Z,\overline{T} \rangle$
                existiert.
                \begin{itemize}
                    \item $\Rightarrow$: \\
                        Sei $T' = \{t_1, \dotsc, t_n\}$ ($t_i = (x_i, y_i,
                        z_i)$) ein \textsc{3D-Matching}.  $\overline{T'}$ werde
                        wie folgt konstruiert:
                        \begin{equation*}
                            \overline{T'} = \{ (w_i, x_i, y_i, z_i)\ |\ i = 1, \dotsc, n \}
                        \end{equation*}
                        Man sieht, dass $\underline{T'}$ ein
                        \textsc{4D-Matching} bildet, da jedes Tupel aus einem
                        der Tripel aus $T'$ sowie einem Element aus $W$
                        zusammengesetzt wird, wobei jedes $w_i \in W$ exakt
                        einmal verwendet wird.  Diese Konstruktion ist möglich,
                        da in $\overline{T}$ jedes Tripel mit jedem $w_i$
                        kombiniert wurde.

                    \item $\Leftarrow$: \\
                        Sei $\overline{T'}$ ein \textsc{4D-Matching} für das
                        konstruierte Problem $\langle W,X,Y,Z,\overline{T}
                        \rangle$.
                        \begin{equation*}
                            T' = \left\{ (x,y,z)\ |\ (w,x,y,z) \in \overline{T'} \right\}
                        \end{equation*}
                        ist dann offensichtlich ein \textsc{3D-Matching} für
                        das ursprüngliche Problem. (Die geforderte Eigenschaft
                        bleibt für die restlichen drei Mengen bestehen, wenn
                        die erste Komponente weggelassen wird.)
                \end{itemize}

                Es gilt also \textsc{3D-Matching} $\leq_p$
                \textsc{4D-Matching}, letzteres ist also mindestens so schwer
                wie ersteres. Da wir von \textsc{3D-Matching} wissen, dass es
                $NP$-vollständig ist, folgt, dass \textsc{4D-Matching}
                $NP$-schwierig ist.

        \end{enumerate}
        Aus i. und ii. folgt, dass \textsc{4D-Matching} $NP$-vollständig ist.
    \end{proof}

    \item
        \begin{proposition}
            \textsc{$k$-Clique} ist für kein festes $k \geq 1$
            $NP$-vollständig, solange $P \neq NP$.
        \end{proposition}
        \begin{proof}
            Wir zeigen, dass \textsc{$k$-Clique} $\in P$, indem wir einen
            Algorithmus angeben, der \textsc{$k$-Clique} in Polynomialzeit
            entscheidet.
            Sei $n = |V|$.
            Dann gibt es $\binom{n}{k} \in \Oh(n^k)$ Subgraphen der Größe $k$.
            In jedem dieser Subgraphen gibt es $\Oh(k^2)$ Kanten, die auf
            Existenz überprüft werden müssen.
            Damit kann \textsc{$k$-Clique} in $\Oh(n^kk^2)$ entschieden werden,
            also in Polynomialzeit, da $k$ fix ist.

            Damit ist \textsc{$k$-Clique} nicht $NP$-vollständig, solange $P \neq NP$.
        \end{proof}

\end{enumerate}

\item % 2.
\begin{enumerate}
    \item
        \begin{proposition}
            \textsc{Hitting Set} ist $NP$-vollständig.
        \end{proposition}
        \begin{proof}
            \hfill \\
            \begin{enumerate}
                \item \textsc{Hitting Set} $\in NP$: \\
                    Gegeben ein Hitting Set $H$ als Zertifikat, lässt sich in
                    polynomieller Zeit prüfen, ob $H \subseteq A$ und $H \cap
                    B_i \neq \emptyset$, $i = 1, \dotsc, m$ gelten.

                \item \textsc{Hitting Set} ist $NP$-schwierig: \\
                    Wir zeigen dies durch die Reduktion von 3-SAT.
                    Sei $\phi$ eine Eingabe für 3-SAT. Wir wählen $A$ als die
                    Menge aller (positiver und negativer) Literale für die
                    Atome in $\phi$.
                    Sei $n$ die Anzahl der Atome. Für jedes Atom $a_i$ in
                    $\phi$ setzen wir $B_i = \{a_i, \lnot a_i\}$, $i = 1,
                    \dotsc, n$. Weiterhin seien $B_{n+1}, \dotsc, B_{n+m}$ die
                    $m$ Klauseln von $\phi$.
                    Diese Reduktion ist ebenfalls effizient möglich, da linear
                    in $n+m$ Mengen erstellt werden müssen, die jeweils eine
                    konstante Anzahl an Elementen beinhalten.

                    Wir zeigen nun, dass $\phi$ genau dann eine Ja-Eingabe für
                    3-SAT ist, wenn die Literale mit wahrem Wahrheitswert einer
                    erfüllenden Belegung von $\phi$ ein Hitting Set für $k = n$
                    und obige Mengen ist.
                    \begin{itemize}
                        \item $\Leftarrow$: \\
                            Da $k = n$ gilt und $B_1, \dotsc, B_n$ jeweils ein
                            Literalpaar eines Atoms enthalten, muss im Hitting
                            Set genau ein Literal für jedes Atom vorkommen.
                            Weiterhin muss aus jeder der Klauselmengen
                            $B_{n+1}, \dotsc, B_{n+m}$ mindestens ein Literal
                            im Hitting Set vorkommen. Nimmt man die Literale im
                            Hitting Set als wahr an und die restlichen als
                            false, so ist leicht zu sehen, dass dies eine
                            erfüllende Belegung für $\phi$ ist, da alle
                            Klauseln erfüllt sind.

                        \item $\Rightarrow$: \\
                            Sei nun eine erfüllende Belegung für $\phi$
                            gegeben. Nach Konstruktion trifft die Menge der
                            wahren Literale die Mengen $B_1, \dotsc, B_n$ (ein
                            Atom ist wahr oder falsch). Dann gibt es in jeder
                            Klausel mindestens ein wahres Literal, die Mengen
                            $B_{n+1}, \dotsc, B_{n+m}$ werden also auch
                            getroffen. Es handelt sich um ein Hitting Set der
                            Größe n.
                    \end{itemize}
                    Damit ist gezeigt, dass 3-SAT $\leq_P$ \textsc{Hitting Set}
                    ist, damit ist letzteres $NP$-schwierig.

            \end{enumerate}
            Aus i. und ii. folgt, dass \textsc{Hitting Set} $NP$-vollständig ist.
        \end{proof}

    \item
        \begin{proposition}
            Es gilt \textsc{Vertex Cover} $\leq_P$ \textsc{Hitting Set}.
        \end{proposition}
        \begin{proof}
            Wir zeigen die Behauptung durch eine Reduktion.  Sei eine Eingabe
            für \textsc{Vertex Cover} durch $G = (V, E)$ und $k \geq 1$
            gegeben. Wir konstruieren eine Eingabe für \textsc{Hitting Set} wie
            folgt: Wir setzen $A = V$ und $B_i = \{u_i, v_i\}$ für jede Kante
            $e_i = \{u_i, v_i\} \in E$. $k$ bleibt gleich.
            Da die Eingabe von \textsc{Vertex Cover} quasi übernommen wird, ist
            dies in Polynomialzeit möglich.
            Zu zeigen ist, dass eine Eingabe für \textsc{Vertex Cover} genau
            dann eine Ja-Eingabe ist, wenn die konstruierte Eingabe eine
            Ja-Eingabe für \textsc{Hitting Set} ist.
            \begin{itemize}
                \item $\Rightarrow$: \\
                    Gegeben sei eine Knotenüberdeckung $V' \subseteq V$ von
                    $G$. Dann ist $V'$ auch ein Hitting Set, da zu jeder Kante
                    ein inzidenter Knoten in $V'$ enthalten ist und somit jedes
                    $B_i$ getroffen wird.

                \item $\Leftarrow$: \\
                    Sei $V' \subset A = V$ ein Hitting Set für diese
                    Konstruktion. Nach Definition der $B_i$ ist für jede Kante
                    $e$ mindestens einer der inzidenten Knoten in $V'$
                    enthalten. $V'$ bildet also ein \textsc{Vertex Cover}.
            \end{itemize}
            Damit ist die Behauptung gezeigt.
        \end{proof}

\end{enumerate}

\end{enumerate}


\end{document}

