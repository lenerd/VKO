\documentclass[a4paper]{scrartcl}

% font/encoding packages
\usepackage[utf8]{inputenc}
\usepackage[T1]{fontenc}
\usepackage{lmodern}
\usepackage[ngerman]{babel}
\usepackage[ngerman=ngerman-x-latest]{hyphsubst}

\usepackage{amsmath, amssymb, amsfonts, amsthm}
\usepackage{mathtools}
\usepackage{array}
\usepackage{stmaryrd}
\usepackage{marvosym}
\usepackage{subcaption}
\allowdisplaybreaks{}
\usepackage[output-decimal-marker={,}]{siunitx}
\usepackage[shortlabels]{enumitem}
\usepackage[section]{placeins}
\usepackage{float}
\usepackage{units}
\usepackage{listings}
\usepackage{pgfplots}
\pgfplotsset{compat=1.12}
\usepackage{tikz}
\usepackage{tikz-qtree}
\usetikzlibrary{arrows,automata}
\usepackage{pdflscape}
\usepackage{xcolor}
\definecolor{light-gray}{HTML}{cccccc}
\usepackage{algorithm}
\usepackage{algpseudocode}
\usepackage{hhline}

\newcommand{\approxalg}[1]{$#1$-Ap\-pro\-xi\-ma\-ti\-ons\-al\-go\-rith\-mus}

\newtheorem*{proposition}{Behauptung}
\newtheorem*{definition}{Definition}
\newcommand{\gdw}{\ \Leftrightarrow\ }
\newcommand{\N}{\mathbb{N}}
\newcommand{\Oh}{\mathcal{O}}
\DeclareMathOperator{\im}{im}

\usepackage{fancyhdr}
\pagestyle{fancy}

\lstset{%
    frame=single,
    numbers=left,
    keepspaces,
    language=R,
    title=Listing: \lstname,
}

\def \blattnr {7}

\lhead{VKO -- Blatt {\blattnr}}
\rhead{Alina Bombeck, Lennart Braun, Carolin Konietzny, Tronje Krabbe}
\cfoot{\thepage}


\title{Vertiefung Kombinatorische Optimierung}
\subtitle{Blatt {\blattnr} Hausaufgaben}
\author{%
    Alina Bombeck, 6535392 (Gruppe 1) \and
    Lennart Braun, 6523742 (Gruppe 1) \and
    Carolin Konietzny, 6523939 (Gruppe 1) \and
    Tronje Krabbe, 6435002 (Gruppe 3)
}
\date{zum 6. Juni 2016}
\usepackage{pdfpages} 

\begin{document}
\maketitle


\begin{enumerate}[label=\bfseries \arabic*.]
\item % 1.
    TODO

\item % 2.
\begin{enumerate}
    \item % a)
        \begin{tabular}[t]{c||c|c|c|c|c|c|c|c|c|c|}
            4 & 0 & 2 & 2 & 3 & 3 & 3 & 5 & 7 & 7 & \underline{8} \\ \hline
            3 & 0 & 2 & 2 & 3 & 3 & 3 & 5 & 7 & 7 & 8 \\ \hline
            2 & 0 & 0 & 1 & 1 & 1 & 1 & 5 & 5 & 6 & 6 \\ \hline
            1 & 0 & 0 & 0 & 0 & 0 & 0 & 5 & 5 & 5 & 5 \\ \hline
            0 & 0 & 0 & 0 & 0 & 0 & 0 & 0 & 0 & 0 & 0 \\ \hhline{=#=|=|=|=|=|=|=|=|=|=|}
              & 0 & 1 & 2 & 3 & 4 & 5 & 6 & 7 & 8 & 9\\
        \end{tabular}
        \\ \\
        Gegenstände 1, 2 und 3 werden eingepackt. Das erreicht Gewicht ist 9,
        und der erreichte Wert ist 8.

    \item % b)
        \begin{tabular}[t]{c||c|c|c|c|c|c|c|c|c|c|c|c|}
            4 & 0 & 1 & 1 & 3 & 6 & 6 & 7 & 7 & \underline{9} & 12 & 12 & 14 \\ \hline
            3 & 0 & 1 & 1 & 3 & 6 & 6 & 7 & 7 & \underline{9} & $\infty$ & $\infty$ & $\infty$ \\ \hline
            2 & 0 & 2 & 6 & 6 & 6 & 6 & \underline{8} & $\infty$ & $\infty$ & $\infty$ & $\infty$ & $\infty$ \\ \hline
            1 & 0 & 6 & 6 & 6 & 6 & \underline{6} & $\infty$ & $\infty$ & $\infty$ & $\infty$ & $\infty$ & $\infty$ \\ \hline
            0 & 0 & $\infty$ & $\infty$ & $\infty$ & $\infty$ & $\infty$ & $\infty$ & $\infty$ & $\infty$ & $\infty$ & $\infty$ & $\infty$ \\ \hhline{=#=|=|=|=|=|=|=|=|=|=|=|=|}
              & 0 & 1 & 2 & 3 & 4 & 5 & 6 & 7 & 8 & 9 & 10 & 11 \\
        \end{tabular}
        \\ \\
        Gegenstände 1, 2 und 3 werden eingepackt. Das erreicht Gewicht ist 9,
        und der erreichte Wert ist 8.

\end{enumerate}
\end{enumerate}
\end{document}
