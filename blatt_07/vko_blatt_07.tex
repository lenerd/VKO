\documentclass[a4paper]{scrartcl}

% font/encoding packages
\usepackage[utf8]{inputenc}
\usepackage[T1]{fontenc}
\usepackage{lmodern}
\usepackage[ngerman]{babel}
\usepackage[ngerman=ngerman-x-latest]{hyphsubst}

\usepackage{amsmath, amssymb, amsfonts, amsthm}
\usepackage{mathtools}
\usepackage{array}
\usepackage{stmaryrd}
\usepackage{marvosym}
\usepackage{subcaption}
\allowdisplaybreaks{}
\usepackage[output-decimal-marker={,}]{siunitx}
\usepackage[shortlabels]{enumitem}
\usepackage[section]{placeins}
\usepackage{float}
\usepackage{units}
\usepackage{listings}
\usepackage{pgfplots}
\pgfplotsset{compat=1.12}
\usepackage{tikz}
\usepackage{tikz-qtree}
\usetikzlibrary{arrows,automata}
\usepackage{pdflscape}
\usepackage{xcolor}
\definecolor{light-gray}{HTML}{cccccc}
\usepackage{algorithm}
\usepackage{algpseudocode}
\usepackage{hhline}

\newcommand{\approxalg}[1]{$#1$-Ap\-pro\-xi\-ma\-ti\-ons\-al\-go\-rith\-mus}

\newtheorem*{proposition}{Behauptung}
\newtheorem*{definition}{Definition}
\newcommand{\gdw}{\ \Leftrightarrow\ }
\newcommand{\N}{\mathbb{N}}
\newcommand{\Oh}{\mathcal{O}}
\DeclareMathOperator{\im}{im}

\usepackage{fancyhdr}
\pagestyle{fancy}

\lstset{%
    frame=single,
    numbers=left,
    keepspaces,
    language=R,
    title=Listing: \lstname,
}

\def \blattnr {7}

\lhead{VKO -- Blatt {\blattnr}}
\rhead{Alina Bombeck, Lennart Braun, Carolin Konietzny, Tronje Krabbe}
\cfoot{\thepage}


\title{Vertiefung Kombinatorische Optimierung}
\subtitle{Blatt {\blattnr} Hausaufgaben}
\author{%
    Alina Bombeck, 6535392 (Gruppe 1) \and
    Lennart Braun, 6523742 (Gruppe 1) \and
    Carolin Konietzny, 6523939 (Gruppe 1) \and
    Tronje Krabbe, 6435002 (Gruppe 3)
}
\date{zum 6. Juni 2016}
\usepackage{pdfpages} 

\begin{document}
\maketitle


\begin{enumerate}[label=\bfseries \arabic*.]
\item % 1.
    Wir formulieren das Hitting Set Problem als Integer Programming Problem
    (IP-Problem):
    \begin{equation*}
        \begin{gathered}
            \text{minimiere } \sum_{a_i \in A} x_i \cdot w_i \\
            \text{unter den Bedingungen:} \\
            \sum_{a_i \in B_j} x_i \geq 1 \quad j \in \{1, \dotsc, m\} \\
            x_i \in \{0,1\} \quad i \in \{1, \dotsc, n\} \\
        \end{gathered}
    \end{equation*}
    Eine optimale Lösung des IP-Problems ist eine optimale Lösung des Hitting
    Set Problems und umgekehrt. Die Variablen $x_i$ beschreiben das
    Enthaltensein des dazugehörigen Elements $a_i$ im Hitting Set $H$.
    \begin{equation*}
        x_i =
        \begin{cases}
            0 & a_i \notin H \\
            1 & a_i \in H \\
        \end{cases}
    \end{equation*}
    Der Wertebereich der Variablen ist $\{0,1\}$, da jedes Element entweder in
    $H$ enthalten ist oder nicht. Die weiteren Bedingungen sorgen dafür, dass
    aus jeder Teilmenge $B_j$ mindestens ein Element auch in $H$ enthalten ist.

    Wir wenden nun LP-Relaxation auf das Problem an. D.\,h. wir betrachten
    modifizieren obiges IP-Problem, so dass die Variablen aus den
    nichtnegativen reellen Zahlen gewählt werden.
    \begin{equation*}
        \begin{gathered}
            \text{minimiere } \sum_{a_i \in A} x_i \cdot w_i \\
            \text{unter den Bedingungen:} \\
            \sum_{a_i \in B_j} x_i \geq 1 \quad j \in \{1, \dotsc, m\} \\
            x_i \geq 0 \quad i \in \{1, \dotsc, n\} \\
        \end{gathered}
    \end{equation*}
    Sei nun $x^\ast = (x^\ast_1, \dotsc, x^\ast_n)$ eine optimale Lösung dieses
    LP-Problems.
    Dann sei $S = \{ a_i \mid x^\ast_i \geq \frac{1}{b} \}$.
    Wir zeigen nun, dass $S$ ein Hitting Set ist und höchstens das $b$-fache
    Gewicht eines optimalen Hitting Sets besitzt.
    (Es war $b = \max\{|B_j| : 1 \leq j \leq m\}$)

    Da $x^\ast$ eine Lösung für das LP-Problem ist, gilt $\sum_{a_i \in B_j}
    x^\ast_i \geq 1$ für jedes $B_j$.  Da $|B_j| \leq b$, muss es mindestens
    einen Summanden $x^\ast_i \geq \frac{1}{b}$ in dieser Summe geben.
    Es ist also $S \cap B_j \neq \emptyset$, $j \in \{1, \dotsc, m\}$ und $S$
    ein Hitting Set.

    Sei $S^\ast$ ein Hitting Set mit minimalem Gewicht. Dann gilt
    \begin{equation*}
        \begin{gathered}
            \sum_{a_i \in S^\ast} w_i
            \geq \sum_{a_i \in A} x^\ast_i \cdot w_i
            \geq \sum_{a_i \in S} x^\ast_i \cdot w_i
            \geq \sum_{a_i \in S} \frac{1}{b} \cdot w_i
            \geq \frac{1}{b} \cdot \sum_{a_i \in S} w_i \\
            \gdw \\
            \frac{\sum_{a_i \in S} w_i}{\sum_{a_i \in S^\ast} w_i}
            \leq b.
        \end{gathered}
    \end{equation*}
    Die beschriebene Methode ist also ein \approxalg{b} für das Hitting Set
    Problem.

\item % 2.
    Die gegebene Eingabe ist
    \begin{table}[H]
        \centering
        \begin{tabular}{c||c|c|c|c}
            Item   & 1 & 2 & 3 & 4 \\ \hline \hline
            Weight & 6 & 2 & 1 & 5 \\ \hline
            Value  & 5 & 1 & 2 & 3 \\
        \end{tabular}
    \end{table}
    mit einer Gewichtsbeschränkung von $W = 9$.

    Das Problem wurde mit den Algorithmen Dynamic Programming 1 und 2 gelöst
    (siehe Tabellen \ref{tab:dp1} und \ref{tab:dp2}).
    Das Ergebnis ist jeweils die Auswahl der Items 1, 2 und 3 mit Gesamtgewicht
    9 und erreichtem Wert 8.

    \begin{table}[H]
        \centering
        \begin{tabular}[t]{c||c|c|c|c|c|c|c|c|c|c}
            4 & 0 & 2 & 2 & 3 & 3 & 3 & 5 & 7 & 7 & \underline{8} \\ \hline
            3 & 0 & 2 & 2 & 3 & 3 & 3 & 5 & 7 & 7 & \underline{8} \\ \hline
            2 & 0 & 0 & 1 & 1 & 1 & 1 & 5 & 5 & \underline{6} & 6 \\ \hline
            1 & 0 & 0 & 0 & 0 & 0 & 0 & \underline{5} & 5 & 5 & 5 \\ \hline
            0 & 0 & 0 & 0 & 0 & 0 & 0 & 0 & 0 & 0 & 0 \\ \hhline{=#=|=|=|=|=|=|=|=|=|=|}
              & 0 & 1 & 2 & 3 & 4 & 5 & 6 & 7 & 8 & 9\\
        \end{tabular}
        \caption{Tabelle für Dynamic Programming 1}
        \label{tab:dp1}
    \end{table}

    \begin{table}[H]
        \centering
        \begin{tabular}[t]{c||c|c|c|c|c|c|c|c|c|c|c|c}
            4 & 0 & 1 & 1 & 3 & 6 & 6 & 7 & 7 & \underline{9} & 12 & 12 & 14 \\ \hline
            3 & 0 & 1 & 1 & 3 & 6 & 6 & 7 & 7 & \underline{9} & $\infty$ & $\infty$ & $\infty$ \\ \hline
            2 & 0 & 2 & 6 & 6 & 6 & 6 & \underline{8} & $\infty$ & $\infty$ & $\infty$ & $\infty$ & $\infty$ \\ \hline
            1 & 0 & 6 & 6 & 6 & 6 & \underline{6} & $\infty$ & $\infty$ & $\infty$ & $\infty$ & $\infty$ & $\infty$ \\ \hline
            0 & 0 & $\infty$ & $\infty$ & $\infty$ & $\infty$ & $\infty$ & $\infty$ & $\infty$ & $\infty$ & $\infty$ & $\infty$ & $\infty$ \\ \hhline{=#=|=|=|=|=|=|=|=|=|=|=|=|}
              & 0 & 1 & 2 & 3 & 4 & 5 & 6 & 7 & 8 & 9 & 10 & 11 \\
        \end{tabular}
        \caption{Tabelle für Dynamic Programming 2}
        \label{tab:dp2}
    \end{table}

\end{enumerate}
\end{document}
