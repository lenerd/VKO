\documentclass[a4paper]{scrartcl}

% font/encoding packages
\usepackage[utf8]{inputenc}
\usepackage[T1]{fontenc}
\usepackage{lmodern}
\usepackage[ngerman]{babel}
\usepackage[ngerman=ngerman-x-latest]{hyphsubst}

\usepackage{amsmath, amssymb, amsfonts, amsthm}
\usepackage{mathtools}
\usepackage{array}
\usepackage{stmaryrd}
\usepackage{marvosym}
\usepackage{subcaption}
\allowdisplaybreaks{}
\usepackage[output-decimal-marker={,}]{siunitx}
\usepackage[shortlabels]{enumitem}
\usepackage[section]{placeins}
\usepackage{float}
\usepackage{units}
\usepackage{listings}
\usepackage{pgfplots}
\pgfplotsset{compat=1.12}
\usepackage{tikz}
\usepackage{tikz-qtree}
\usetikzlibrary{arrows,automata}
\usepackage{pdflscape}
\usepackage{xcolor}
\definecolor{light-gray}{HTML}{cccccc}
\usepackage{algorithm}
\usepackage{algpseudocode}
\usepackage{hhline}

\newcommand{\approxalg}[1]{$#1$-Ap\-pro\-xi\-ma\-ti\-ons\-al\-go\-rith\-mus}

\newtheorem*{proposition}{Behauptung}
\newtheorem*{definition}{Definition}
\newcommand{\gdw}{\ \Leftrightarrow\ }
\newcommand{\N}{\mathbb{N}}
\newcommand{\Oh}{\mathcal{O}}
\DeclareMathOperator{\im}{im}

\usepackage{fancyhdr}
\pagestyle{fancy}

\lstset{%
    frame=single,
    numbers=left,
    keepspaces,
    language=R,
    title=Listing: \lstname,
}

\def \blattnr {8}

\lhead{VKO -- Blatt {\blattnr}}
\rhead{Alina Bombeck, Lennart Braun, Carolin Konietzny, Tronje Krabbe}
\cfoot{\thepage}


\title{Vertiefung Kombinatorische Optimierung}
\subtitle{Blatt {\blattnr} Hausaufgaben}
\author{%
    Alina Bombeck, 6535392 (Gruppe 1) \and
    Lennart Braun, 6523742 (Gruppe 1) \and
    Carolin Konietzny, 6523939 (Gruppe 1) \and
    Tronje Krabbe, 6435002 (Gruppe 3)
}
\date{zum 20. Juni 2016}
\usepackage{pdfpages} 

\begin{document}
\maketitle


\begin{enumerate}[label=\bfseries \arabic*.]
\item % 1.
    Gegeben: $ c(M) \leq \frac{1}{2} c(H^*)$.\\
    Daraus können wir folgern:
    \begin{equation*}
        c(H) \leq c(T^+) = c(L) = c(M) + c(T) \leq \frac{1}{2} c(H^*) + c(H^*) = \frac{3}{2} c(H^*)
        \qed
    \end{equation*}
    Da man in Schritt (4) $L$ durchläuft, und seine Knoten ohne Duplikate
    wählt, gilt $c(H) \leq c(L)$. $L$ ergibt sich durch die Vereinigung von
    $M$ und $T$, und $c(M) \leq \frac{1}{2}c(H^*)$ folgt aus der Annahme
    ($\star$). $c(T) \leq c(H^*)$ ergibt sich daraus, dass, nimmt
    man eine Kante aus $H^*$ weg, man einen minimalen aufspannenden Baum
    erhält.

\item % 2.
    Wir nutzen einen Greedy-Algorithmus:\\
    Nimm so lange ein beliebiges Tripel, bis es nicht mehr geht.
    Dann passt das schon. $\qed$

\end{enumerate}
\end{document}
