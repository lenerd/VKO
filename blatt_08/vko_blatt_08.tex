\documentclass[a4paper]{scrartcl}

% font/encoding packages
\usepackage[utf8]{inputenc}
\usepackage[T1]{fontenc}
\usepackage{lmodern}
\usepackage[ngerman]{babel}
\usepackage[ngerman=ngerman-x-latest]{hyphsubst}

\usepackage{amsmath, amssymb, amsfonts, amsthm}
\usepackage{mathtools}
\usepackage{array}
\usepackage{stmaryrd}
\usepackage{marvosym}
\usepackage{subcaption}
\allowdisplaybreaks{}
\usepackage[output-decimal-marker={,}]{siunitx}
\usepackage[shortlabels]{enumitem}
\usepackage[section]{placeins}
\usepackage{float}
\usepackage{units}
\usepackage{listings}
\usepackage{pgfplots}
\pgfplotsset{compat=1.12}
\usepackage{tikz}
\usepackage{tikz-qtree}
\usetikzlibrary{arrows,automata}
\usepackage{pdflscape}
\usepackage{xcolor}
\definecolor{light-gray}{HTML}{cccccc}
\usepackage{algorithm}
\usepackage{algpseudocode}
\usepackage{hhline}

\newcommand{\approxalg}[1]{$#1$-Ap\-pro\-xi\-ma\-ti\-ons\-al\-go\-rith\-mus}

\newtheorem*{proposition}{Behauptung}
\newtheorem*{definition}{Definition}
\newcommand{\gdw}{\ \Leftrightarrow\ }
\newcommand{\N}{\mathbb{N}}
\newcommand{\Oh}{\mathcal{O}}
\DeclareMathOperator{\im}{im}

\usepackage{fancyhdr}
\pagestyle{fancy}

\lstset{%
    frame=single,
    numbers=left,
    keepspaces,
    language=R,
    title=Listing: \lstname,
}

\def \blattnr {8}

\lhead{VKO -- Blatt {\blattnr}}
\rhead{Alina Bombeck, Lennart Braun, Carolin Konietzny, Tronje Krabbe}
\cfoot{\thepage}


\title{Vertiefung Kombinatorische Optimierung}
\subtitle{Blatt {\blattnr} Hausaufgaben}
\author{%
    Alina Bombeck, 6535392 (Gruppe 1) \and
    Lennart Braun, 6523742 (Gruppe 1) \and
    Carolin Konietzny, 6523939 (Gruppe 1) \and
    Tronje Krabbe, 6435002 (Gruppe 3)
}
\date{zum 20. Juni 2016}
\usepackage{pdfpages}

\begin{document}
\maketitle


\begin{enumerate}[label=\bfseries \arabic*.]
\item % 1.
    \begin{proposition}
        Der gegebene Algorithmus ist ein \approxalg{\frac{3}{2}} für $\Delta$-TSP.
    \end{proposition}
    \begin{proof}
        Zunächst gehen wir davon aus, dass $(\star)\ c(M) \leq \frac{1}{2} c(H^*)$ gilt.
        Daraus können wir folgern
        \begin{equation*}
            c(H)
            \leq c(T^+) = c(L) = c(M) + c(T)
            \stackrel{(\star)}{\leq} \frac{1}{2} c(H^*) + c(H^*) = \frac{3}{2} c(H^*),
        \end{equation*}
        denn $c(H) \leq c(T^+) = c(L)$ folgt aus der Dreiecksungleichung, da man
        bei der Eulertour abkürzt, und die Kosten des MSTs $T$ geringer sein
        müssen, als die der optimalen TSP-Tour (sonst gäbe es einen billigeren
        MST).

        Nun zeigen wir, dass $(\star)$ gilt.
        Sei $\tilde{H}^\ast$ ein optimaler Hamiltonkreis in $G'$. Da $M$ ein
        Teil von $\tilde{H}^\ast$ ist (andernfalls wäre $M$ nicht minimal) und
        die Dreiecksungleichung gilt, folgt
        $c(M) \leq  \frac{1}{2}c(\tilde{H}^\ast) \leq \frac{1}{2}c(H^\ast)$.
    \end{proof}

\item % 2.
    Wir nutzen einen Greedy-Algorithmus:\\
    Nimm so lange ein beliebiges Tripel, bis es nicht mehr geht.
    Dann passt das schon:
    \begin{algorithm}[H]
        \begin{algorithmic}[1]
            \Procedure{GreedyMax3DMatching}{$T,X,Y,Z$}
            \State $M \gets \emptyset$
            \State $U \gets \emptyset$
            \For {$(x,y,z) \in T$}
            \If {$x,y,z \notin U$}
                \State $M \gets M \cup \{(x,y,z)\}$
                \State $U \gets U \cup \{x,y,z\}$
            \EndIf
            \EndFor
            \State \Return $M$
            \EndProcedure
        \end{algorithmic}
    \end{algorithm}

    Der Algorithmus iteriert über die Tripel aus $T$ in beliebiger Reihenfolge.
    Passt ein $(x,y,z) \in T$ noch in das Matching, so wird es zu $M$ und die
    $x,y,z$ der Menge der benutzten Elemente $U$ hinzugefügt.
    Andernfalls ist eines der $x,y,z$ schon in $U$ enthalten und blockiert
    dieses Tripel.
    Der Algorithmus läuft in Polynomialzeit und das Ergebnis ist ein maximales
    Matching, d.\,h. es kann kein weiteres Element hinzugefügt werden.

    Sei $M^\ast$ ein optimales Matching.
    Da jedes gewählte Tripel in $M$ sich mit höchstens drei Tripeln aus $M^\ast$
    überschneiden (und deren Wahl blockieren) kann, ist $M \geq \frac{1}{3} M^\ast$.

\end{enumerate}
\end{document}
