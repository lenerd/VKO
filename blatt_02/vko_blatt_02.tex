\documentclass[a4paper]{scrartcl}

% font/encoding packages
\usepackage[utf8]{inputenc}
\usepackage[T1]{fontenc}
\usepackage{lmodern}
\usepackage[ngerman]{babel}
\usepackage[ngerman=ngerman-x-latest]{hyphsubst}

\usepackage{amsmath, amssymb, amsfonts, amsthm}
\usepackage{mathtools}
\usepackage{array}
\usepackage{stmaryrd}
\usepackage{marvosym}
\usepackage{subcaption}
\allowdisplaybreaks{}
\usepackage[output-decimal-marker={,}]{siunitx}
\usepackage[shortlabels]{enumitem}
\usepackage[section]{placeins}
\usepackage{float}
\usepackage{units}
\usepackage{listings}
\usepackage{pgfplots}
\pgfplotsset{compat=1.12}
\usepackage{tikz}
\usetikzlibrary{arrows,automata}

\usepackage{xcolor}
\definecolor{light-gray}{HTML}{cccccc}


\newtheorem*{proposition}{Behauptung}
\newtheorem*{definition}{Definition}
\newcommand{\gdw}{\ \Leftrightarrow\ }
\newcommand{\N}{\mathbb{N}}
\newcommand{\Oh}{\mathcal{O}}
\DeclareMathOperator{\im}{im}

\usepackage{fancyhdr}
\pagestyle{fancy}

\lstset{%
    frame=single,
    numbers=left,
    keepspaces,
    language=R,
    title=Listing: \lstname,
}

\def \blattnr {2}

\lhead{VKO -- Blatt {\blattnr}}
\rhead{Alina Bombeck, Lennart Braun, Carolin Konietzny, Tronje Krabbe}
\cfoot{\thepage}


\title{Vertiefung Kombinatorische Optimierung}
\subtitle{Blatt {\blattnr} Hausaufgaben}
\author{%
    Alina Bombeck (Gruppe 1),
    Lennart Braun (Gruppe 1),
    Carolin Konietzny (Gruppe 1),
    Tronje Krabbe (Gruppe 3)
}
\date{zum 25. April 2016}
\usepackage{pdfpages} 

\begin{document}
\maketitle


\begin{enumerate}[label=\bfseries \arabic*.]
\item % 1.
\begin{enumerate}
    \item % a)
    \begin{proposition}
        \[ \text{SET PACKING} \leq_{p} \text{RESOURCE RESERVATION PROBLEM} \]
    \end{proposition}
    \begin{proof}
        \hfill \\
        \begin{enumerate}
            \item Das \text{RESOURCE RESERVATION PROBLEM} liegt in NP. IDese Behauptung wird im Folgenden bewiesen.  
            Man rate eine Ja-Eingabe und zeigt in polynomieller Zeit, dass es sich um eine gültige Lösung handelt, indem man $|Prozesse| = k$ prüft und sicherstellt, dsas die Ressourcenmengen der Prozesse miteinander disjunkt sind.
            \item Das Problem Ressource Reservation Problem ist NP-vollständig, da es nur eine andere Formulierung des Set-Packing-Problems ist, welches nachgewiesenerweise NP-vollständig ist.
        \end{enumerate}
    \end{proof}

    \item % b)
       Wenn k auf 2 gesetzt wird, ist das Problem in P, da man um es zu lösen nur jeweils 2 Mengen auf Disjunktivität vergleichen muss.


    \item % c)
     Das Problem ist eine Umschreibung des Problems der bipartiten Graphen. Das maximale Matching muss hierbei mindestens so groß wie die erforderte Anzahl k der Prozesse erreichen. Ein polynomieller Algorithmus, um dieses Problem zu lösen ist durch Edmond/Karp gegeben.

    \item % d)
        TODO


\end{enumerate}

\item % 2.
\begin{enumerate}
    \item % a)
        TODO
    \item % b)
     NUMBER PARTITION INTO EQUAL PARTS (PARTITION) ist in NP, da es ein Spezialfall von SUBSET SUM ist.
    \begin{proposition}
        \[ \text{SUBSET SUM} \leq_{p} \text{PARTITION} \]
    \end{proposition}
    \begin{proof}
        \hfill \\
        \begin{enumerate}
            \item Eingabe SUBSET SUM: Positive ganze Zahlen $x_1, .... x_n$ , W = Zielgröße
            \item Sei X = $\sum_{i=1}^{n} a_i $
            \item Umformung in Eingabe für PARTITION: $n + 2$ Zahlen, deren Summe den Wert $4X$ ergibt.\\ 
            \begin{align*}
                y_i &= x_i \text{ für } 1 <= i <= n \\
                y_{n+1} &= 2x - W \\
                y_{n+2} &= x + W
            \end{align*}
            TODO BILD % macht Tronje :)

            \item
                \begin{proposition}
                Reduktion ist in polynomieller Zeit berechenbar.
                \end{proposition}
                \begin{proof}
                    \hfill \\
                    ``$\Leftarrow$'' \\
                    Gegeben: eine geeignete Aufteilung der Zahlen für PARTITION.\\
                    So sind $y_{n+1}$ und $y_{n+2}$ nicht in der selben Teilmenge, da
                    $$y_{n+1} + y_{n+2} = 3X$$
                    Hieraus folgt eine Lösung für SUBSET SUM, da sich die Zahlen aus $y_{1,...,n}$,
                    die sich in der selben Teilmenge wie $y_{n+1}$ befinden, \\
                    zu $2X - y_{n+1} = W$ aufsummieren. \\

                    ``$\Rightarrow$'' \\
                    Wenn es eine Teilmenge der Zahlen $x_{1,...,n}$ mit der Summe $W$ gibt, so gibt es auch eine Teilmenge der Zahlen $y_{1,...,n}$
                    mit der Summen $W$.
                    Fügt man die Zahl $y_{n+1} = 2X - W$ zu dieser Teilmenge hinzu, erhalten wir eine Menge, deren Summe $2X$ ist.
                    Dies entspricht der Hälfte der Gesamtmenge.


                \end{proof}

        \end{enumerate}
    \end{proof}
\end{enumerate}

\end{enumerate}


\end{document}

