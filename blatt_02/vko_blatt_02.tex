\documentclass[a4paper]{scrartcl}

% font/encoding packages
\usepackage[utf8]{inputenc}
\usepackage[T1]{fontenc}
\usepackage{lmodern}
\usepackage[ngerman]{babel}
\usepackage[ngerman=ngerman-x-latest]{hyphsubst}

\usepackage{amsmath, amssymb, amsfonts, amsthm}
\usepackage{mathtools}
\usepackage{array}
\usepackage{stmaryrd}
\usepackage{marvosym}
\usepackage{subcaption}
\allowdisplaybreaks{}
\usepackage[output-decimal-marker={,}]{siunitx}
\usepackage[shortlabels]{enumitem}
\usepackage[section]{placeins}
\usepackage{float}
\usepackage{units}
\usepackage{listings}
\usepackage{pgfplots}
\pgfplotsset{compat=1.12}
\usepackage{tikz}
\usetikzlibrary{arrows,automata}

\usepackage{xcolor}
\definecolor{light-gray}{HTML}{cccccc}


\newtheorem*{proposition}{Behauptung}
\newtheorem*{definition}{Definition}
\newcommand{\gdw}{\ \Leftrightarrow\ }
\newcommand{\N}{\mathbb{N}}
\newcommand{\Oh}{\mathcal{O}}
\DeclareMathOperator{\im}{im}

\usepackage{fancyhdr}
\pagestyle{fancy}

\lstset{%
    frame=single,
    numbers=left,
    keepspaces,
    language=R,
    title=Listing: \lstname,
}

\def \blattnr {2}

\lhead{VKO -- Blatt {\blattnr}}
\rhead{Alina Bombeck, Lennart Braun, Carolin Konietzny, Tronje Krabbe}
\cfoot{\thepage}


\title{Vertiefung Kombinatorische Optimierung}
\subtitle{Blatt {\blattnr} Hausaufgaben}
\author{%
    Alina Bombeck (Gruppe 1),
    Lennart Braun (Gruppe 1),
    Carolin Konietzny (Gruppe 1),
    Tronje Krabbe (Gruppe 3)
}
\date{zum 25. April 2016}
\usepackage{pdfpages} 

\begin{document}
\maketitle


\begin{enumerate}[label=\bfseries \arabic*.]
\item % 1.
\begin{enumerate}
    \item % a)
    \begin{proposition}
        \[ \text{SET PACKING} \leq_{p} \text{RESOURCE RESERVATION PROBLEM} \]
    \end{proposition}
    \begin{proof}
        \hfill \\
        \begin{enumerate}
            \item Das \text{RESOURCE RESERVATION PROBLEM} liegt in NP. IDese Behauptung wird im Folgenden bewiesen.  
            Man rate eine Ja-Eingabe und zeigt in polynomieller Zeit, dass es sich um eine gültige Lösung handelt, indem man $|Prozesse| = k$ prüft und sicherstellt, dsas die Ressourcenmengen der Prozesse miteinander disjunkt sind.
            \item Das Problem Ressource Reservation Problem ist NP-vollständig, da es nur eine andere Formulierung des Set-Packing-Problems ist, welches nachgewiesenerweise NP-vollständig ist.
        \end{enumerate}
    \end{proof}

    \item % b)
       Wenn k auf 2 gesetzt wird, ist das Problem in P, da man um es zu lösen nur jeweils 2 Mengen auf Disjunktivität vergleichen muss.


    \item % c)
     Das Problem ist eine Umschreibung des Problems der bipartiten Graphen. Das maximale Matching muss hierbei mindestens so groß wie die erforderte Anzahl k der Prozesse erreichen. Ein polynomieller Algorithmus, um dieses Problem zu lösen ist durch Edmond/Karp gegeben.

    \item % d)
        Der Spezialfall, in dem jede Ressource von höchstens zwei Prozessen
        benötigt wird, ist $NP$-vollständig.
        Dies kann nachgewiesen werden, indem gezeigt wird, dass das Problem zu
        \textsc{Independent Set} äquivalent ist.

        Man kann die Eingabe als Graphen modellieren.
        Jeder Prozess ist ein Knoten und jede Ressource wird durch eine Kante
        zwischen den zwei Knoten dargestellt, die diejenigen Prozesse
        repräsentieren, die von dieser Ressource abhängig sind.
        Ressourcen, die von weniger als zwei Prozess benötigt werden, sind
        irrelevant und können daher ignoriert werden -- sie stehen in jedem
        Fall zur Verfügung, falls benötigt.
        Man sieht nun, dass die Entscheidung, ob eine passende \textsc{Resource
        Reservation} mit $k$ Prozessen existiert, äquivalent zu der
        Entscheidung, ob ein \textsc{Independent Set} mit $k$ Knoten in der
        Graphdarstellung existiert, ist.
        Die Probleme können also aufeinander reduziert werden.

        Wir zeigen eine Reduktion von \textsc{Independent Set} auf diese
        Version von \textsc{Resource Allocation}.
        Gegeben eine Instanz $G = (V,E), k$ für \textsc{Independent Set}.
        Wir setzen $P = V$ und $R = E$, $k$ bleibt gleich.
        Die zwei Knoten $u,v$ in jeder Kante $e = \{u,v\}$ seien die Prozesse,
        die diese Ressource benötigen.

        Zu zeigen ist, dass $G,k$ genau dann eine Ja-Eingabe für
        \textsc{Independent Set} ist, wenn $P,R,k$ eine Ja-Eingabe für
        \textsc{Resource Allocation} ist.
        \begin{itemize}
            \item $\Rightarrow$: \\
                Es gibt $\geq k$ Knoten in $V$, die paarweise keine gemeinsamen
                Kanten in $E$ haben.
                Nach Konstruktion gibt es in $\geq k$ Prozesse in $P$, so dass
                keine Ressource in $R$ von mehr als einem Prozess benötigt
                wird.
            \item $\Leftarrow$: \\
                Es gibt $\geq k$ Prozesse, so dass keine Ressource von mehr als
                einem Prozess benötigt wird.
                Damit gibt es auch $\geq k$ Knoten, die paarweise nicht durch
                Kanten verbunden sind.
        \end{itemize}
        Das Problem ist also mindestens so schwer wie \textsc{Independent Set}
        und damit $NP$-schwierig.

\end{enumerate}

\item % 2.
\begin{enumerate}
    \item % a)
        \begin{proposition}
            \textsc{Set Splitting} ist $NP$-vollständig.
        \end{proposition}
        \begin{proof}
            \hfill \\
            \begin{enumerate}
                \item \textsc{Set Splitting} $\in NP$: \\
                    Gegeben eine Partition $S_1,S_2$ von $S$, so kann effizient
                    mit geeigneten Mengenoperationen überprüft werden, ob die
                    gewünschte Eigenschaft erfüllt ist.
                \item \textsc{Set Splitting} ist $NP$-schwierig: \\
                    Wir zeigen durch eine Reduktion, dass 3-SAT $\leq_p$
                    \textsc{Set Splitting} gilt.
                    Sei eine Klauselmenge $\phi$ eine Eingabe für 3-SAT und $A$
                    die Menge der Atome in $\phi$.
                    Wir setzen $S = A \cup \{\bot\}$. $\bot$ ist das Bottom,
                    dem stets der Wahrheitswert falsch zugewiesen wird. 
                    Jede Klausel $K \in M$ wird mit einem $\bot$ erweitert und
                    in $\mathcal{C}$ eingefügt. Weiterhin bildet jedes Atom $a$
                    zusammen mit seinem negativen Literal $\lnot a$ eine Menge
                    in $\mathcal{C}$.
                    \begin{equation*}
                        \mathcal{C} =
                        \Big\{ K \cup \{\bot\}\ |\ K \in M \Big\}
                        \cup
                        \Big\{ \{a, \lnot a\}\ |\ a \in A \Big\}
                    \end{equation*}
                    Wir zeigen nun, dass $\phi$ genau dann erfüllbar ist, wenn
                    es ein Set Splitting für $S, \mathcal{C}$ gibt.
                    \begin{itemize}
                        \item $\Rightarrow$: \\
                            Sei $\phi$ erfüllbar und $T$ die Menge der wahren
                            Literale in einer erfüllenden Belegung.
                            Dann ist $(T, S \setminus T)$ ein Set Splitting von
                            $S$, da
                            \begin{itemize}
                                \item
                                    für jedes Atom $a$ ist genau ein Literal
                                    $a$ oder $\lnot a$ mit wahr belegt
                                \item
                                    in jeder Klauselmenge ist mindestens ein
                                    Literal wahr, $\bot$ ist nie wahr
                            \end{itemize}
                            Damit ist keine Menge $C \in \mathcal{C}$ komplett
                            in $T$ oder $S \setminus T$ enthalten und dies ist
                            ein Set Splitting.

                        \item $\Leftarrow$: \\
                            Sei $(S_1,S_2)$ ein Set Splitting und dabei sei
                            o.\,B.\,d.\,A. $\bot \in S_2$.
                            Es sind also für jede Atom das positive und das
                            negative Literal in verschiedenen Mengen.
                            Belegt man die Literale in $S_1$ mit wahr, ergibt
                            sich also eine valide Belegung.
                            Weiterhin ist mindestens ein Literal aus jeder
                            Klausel in $S_1$ enthalten (denn $\bot \in S_2$).
                            Damit erhält man auch eine erfüllende Belegung von
                            $\phi$.
                    \end{itemize}
                    3-SAT ist auf \textsc{Set Splitting} reduzierbar.
            \end{enumerate}
            Aus i. und ii. folgt die Behauptung.
        \end{proof}

    \item % b)
     NUMBER PARTITION INTO EQUAL PARTS (PARTITION) ist in NP, da es ein Spezialfall von SUBSET SUM ist.
    \begin{proposition}
        \[ \text{SUBSET SUM} \leq_{p} \text{PARTITION} \]
    \end{proposition}
    \begin{proof}
        \hfill \\
        \begin{enumerate}
            \item Eingabe SUBSET SUM: Positive ganze Zahlen $x_1, \dotsc x_n$ , W = Zielgröße
            \item Sei X = $\sum_{i=1}^{n} a_i $
            \item Umformung in Eingabe für PARTITION: $n + 2$ Zahlen, deren Summe den Wert $4X$ ergibt.\\ 
            \begin{align*}
                y_i &= x_i \text{ für } 1 <= i <= n \\
                y_{n+1} &= 2x - W \\
                y_{n+2} &= x + W
            \end{align*}
            TODO BILD % macht Tronje :)

            \item
                \begin{proposition}
                Reduktion ist in polynomieller Zeit berechenbar.
                \end{proposition}
                \begin{proof}
                    \hfill \\
                    ``$\Leftarrow$'' \\
                    Gegeben: eine geeignete Aufteilung der Zahlen für PARTITION.\\
                    So sind $y_{n+1}$ und $y_{n+2}$ nicht in der selben Teilmenge, da
                    $$y_{n+1} + y_{n+2} = 3X$$
                    Hieraus folgt eine Lösung für SUBSET SUM, da sich die Zahlen aus $y_{1,\dotsc,n}$,
                    die sich in der selben Teilmenge wie $y_{n+1}$ befinden, \\
                    zu $2X - y_{n+1} = W$ aufsummieren. \\

                    ``$\Rightarrow$'' \\
                    Wenn es eine Teilmenge der Zahlen $x_{1,\dotsc,n}$ mit der Summe $W$ gibt, so gibt es auch eine Teilmenge der Zahlen $y_{1,\dotsc,n}$
                    mit der Summen $W$.
                    Fügt man die Zahl $y_{n+1} = 2X - W$ zu dieser Teilmenge hinzu, erhalten wir eine Menge, deren Summe $2X$ ist.
                    Dies entspricht der Hälfte der Gesamtmenge.


                \end{proof}

        \end{enumerate}
    \end{proof}
\end{enumerate}

\end{enumerate}


\end{document}

